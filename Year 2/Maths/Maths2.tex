\documentclass[Maths.tex]{subfiles}

\begin{document}

\part{}
\chapter{}
\section{Introduction}
This part of the course will mainly deal with differential equations.
\subsection{Classes of Differential Equation}
\begin{itemize}
    \item Ordinary Differential Equations, e.g.
        \begin{equation*}
            \frac{dy}{dx} + f(x, y) = 0
        \end{equation*}
    \item Partial Differential Equations, e.g.
        \begin{equation*}
            \frac{\p g(x, y)}{\p x} + \frac{\p g(x, y)}{\p y} = 0
        \end{equation*}
\end{itemize}

\subsection{Order of ODEs}
The order of an ODE is the value of the highest derivative present.
\begin{align}
    \frac{dy}{dx} + y^2 + \sqrt{xy} &= 0 \tag{1st Order} \\
    \frac{d^2y}{dx^2} + \frac{dy}{dx} + f(x, y) &= 0 \tag{2nd Order}
\end{align}

\subsection{Degree of ODEs}
The degree of an ODE is the power of the highest order term after all the derivatives have been rationalised.
\begin{align}
    \bigg(\frac{d^2y}{dx^2}\bigg)^2 + \bigg(\frac{dy}{dx}\bigg)^3 + f(x, y) &= 0 \tag{2nd Degree} \\
    \bigg(\frac{d^2y}{dx}\bigg) + \left(\frac{dy}{dx}\right)^{3/2} + y &= 0 \tag{4th Degree} \\
    \frac{d^2y}{dx^2} + \sqrt{xy} &= 0 \tag{2nd Degree}
\end{align}

\subsection{Solution to ODEs}
\begin{itemize}
    \item The most general function that solves the equation.
    \item "General solution" contains integration constants that are not fixed by the equation.
    \item These constants can be fixed by boundary conditions which leads to a "particular solution".
    \item An $n$th order ODE has $n$ integration constants.
\end{itemize}

\section{1st Order, 1st Degree ODEs}
\begin{equation*}
    \frac{dy}{dx} = f(x,y) \to A(x,y)\,dx + B(x,y)\,dy = 0
\end{equation*}

\subsection{Separable Equations}
\begin{align*}
    f(x,y) &= f(x)g(y) \\
    \dy &= f(x)g(y) \\
    \int \frac{dy}{g(y)} &= \int f(x)\,dx
\end{align*}

\begin{example}
    \begin{align*}
        x^2 \dy &= 1 + y \\
        \dy &= \frac{1 + y}{x^2} = (1 + y) \cdot \frac{1}{x^2} \\
        \int \frac{dy}{1 + y} &= \int \frac{dx}{x^2} \\
        \ln(1 + y) &= -\frac{1}{x} + c \\
        1 + y &= Ae^{-\frac{1}{x}} \\
        y &= Ae^{-\frac{1}{x}} - 1
    \end{align*}
\end{example}

\subsection{Exact Equations}
\begin{align*}
    A(x,y)\,dx &+ B(x,y)\,dy = 0 \\
    \p_yA(x,y) &= \p_xB(x,y) \\
    U(x,y) \to dU &= \p_xU\,dx + \p_yU\,dy \\
    dU = 0 &\to U = c \\
    A(x,y) &= \p_xU \to U(x,y) = \int A(x,y)\,dx + F(y) \\
    B(x,y) &= \p_yU  \\
    \p_yU &= \p_y \left[\int A(x,y)\,dx \right] + F'(y) = B(x,y)
\end{align*}

\begin{example}
    \begin{align*}
        \frac{x}{2}\dy + x^2 + \frac{y}{2} &= 0 \\
        A = x^2 + \frac{y}{2} ~&;~ B = \frac{x}{2} \\
        \p_yA = \frac{1}{2} ~&;~ \p_xB = \frac{1}{2} \implies \text{Exact} \\
        x^2 + \frac{y}{2} = \frac{\p U}{\p x} &\to U(x,y) = \frac{x^3}{3} + \frac{xy}{2} + F(y) \\
        \p_yU(x,y) = \frac{x}{2} + F'(y) = \frac{x}{2} &\implies F'(y) = 0 \to F(y) = c \\
        U(x,y) &= \frac{x^3}{3} + \frac{xy}{2} = d \\
        y &= -\frac{2}{3}x^2 + \frac{2d}{x}
    \end{align*}
\end{example}

\subsection{The Integrating Factor}
\begin{align*}
    A(x,y)\,dx + B(x,y)\,dy &= 0 \\
    \p_yA \neq \p_xB &\to \mu(x,y)A(x,y)\,dx + \mu(x,y)B(x,y)\,dy = 0 \\
    \p_y[\mu A] &= \p_x[\mu B]
\end{align*}
$\mu$ is called the integrating factor
\begin{itemize}
    \item If $\mu = \mu(x)$:
        \begin{align*}
            \p_y[\mu A] &= \mu \p_yA = \mu'B + \mu\p_xB \\
            \frac{d\mu}{\mu} &= \frac{1}{B}(\p_yA - \p_xB) = f(x)
        \end{align*}
    \item If $\mu = \mu(y)$:
        \begin{equation*}
            \frac{d\mu}{\mu} = \frac{1}{A}(\p_xB - \p_yA) = g(y)
        \end{equation*}
    \item Special case: linear equations
        \begin{align*}
            \dy + P(x)y &= Q(x) \\
            A = P(x)y - Q ~&;~ B = 1 \\
            \frac{1}{B}(\p_yA - \p_xB) &= \frac{1}{1}(P - 0) = P(x) \\
            \frac{d\mu}{\mu} = P(x) &\to \mu = e^{\int P(x)\,dx}
        \end{align*}
\end{itemize}

\begin{example}
    \begin{align*}
        \dy + xy + x^2 &= 0 \\
        dy + \left(\frac{y}{x} + x^2\right)\,dx &= 0 \\
        \mu = e^{\int \frac{dx}{x}} &= x \\
        x\,dy + (y + x^3)\,dx &= 0 \\
        \p_xB = 1 ~;~ \p_yA &= 1
    \end{align*}
\end{example}

\chapter{}

\section{Simplifying Equations by Change of Variables}
\subsection{Homogeneous Equations}
\begin{align*}
    \dy &= f\left(\frac{y}{x}\right), \to y = v\cdot x \\
    y' &= v'x + v = f(v) \implies \frac{dx}{x} = \frac{dv}{f(v) - v} \\
    f(x, y) &= \frac{A(x,y)}{B(x,y)} \to \begin{cases} A(\lambda x, \lambda y) &= \lambda^{n}A(x,y) \\
    B(\lambda x, \lambda y) &= \lambda^n B(x,y) \end{cases}
\end{align*}

\begin{example}
\begin{align*}
    xy\dy &+ 3x^2 - y^2 = 0 \\
    \dy &= \frac{y^2 - 3x^2}{xy} \to y = vx \\
    xv' + v &= \frac{v^2x^2 - 3x^2}{vx^2} + v = \frac{v^2 - 3}{v} + v = -\frac{3}{v} \\
    \frac{dv}{dx}x &= -\frac{3}{v} \to v\,dv = -3\frac{dx}{x} \\
    \frac{v^2}{2} &= -3\ln{x} + c \to v^2 = d - 6\ln{x} \\
    v &= \pm \sqrt{d - 6\ln{x}}
\end{align*}
\end{example}

\subsection{Isobaric Equations}
\begin{itemize}
    \item Give $x\,dx$ weight 1
    \item Give $y\,dy$ weight m
    \item If everywhere is the same power, again separable: $y = vx^m$
\end{itemize}

\begin{align*}
    (\underbrace{1}_{0} &+ \underbrace{xy}_{1\;m})\underbrace{dy}_{m} + \underbrace{y^2}_{2m}\underbrace{dx}_{1} = 0 \\
    m &= 2m + 1 \to m = -1 \to y = \frac{v}{x} \\
    \dy &= \frac{v'}{x} - \frac{1}{x^2}v \\
    (1 &+ v)\left(\frac{v'}{x} - \frac{1}{x^2}v\right) + \frac{v^2}{x^2} = 0 \\
    \frac{v'}{x}(1 + v) &= \frac{v}{x^2} \to dv\left(\frac{1}{v} + 1\right) = \frac{dx}{x} \\
    \ln{v} + v &= \ln{x} + c \to \ln{y} + \cancel{\ln{x}} + xy = \cancel{\ln{x}} + c \\
    \ln{y} &+ xy = c
\end{align*}

\subsection{Bernoulli Equation}
\begin{align*}
    \dy &+ P(x)y = Q(x)y^n \to v = y^{1 - n} \\
    v' &= (1 - n)y^{-n} = (1 - n)y^{-n}\left[Q(x)y^n - P(x)y\right] \\
    &= (1-n)Q(x) - P(x)(1 - n) \times y^{1-n} \therefore \text{Linear}
\end{align*}

\section{Linear Higher Order ODEs}
\begin{align*}
    a_n(x)\frac{d^ny}{dx^n} &+ a_{n - 1}(x)\frac{d^{n-1}y}{dx^{n-1}} + \cdots + a_1(x)\dy + a_0(x)y = f(x) \\
    f(x) & \begin{cases} = 0 & \text{homogeneous} \\ \neq 0 & \text{inhomogeneous} \end{cases}
\end{align*}

\begin{itemize}
    \item General solution will have n integration constants
    \item There are n independent solutions
    \item To solve:
    \begin{enumerate}
        \item Set $f(x) = 0$ to get the complementary equation
        \item Solve the complementary equation for n independent solutions
        \item Most general solution, $\{y_i\}$:
            \begin{equation*}
                y_c = c_1y_1 + c_2y_2 + \cdots + c_ny_n
            \end{equation*}
            You will have n linearly independent solutions
        \item $\{y_i\}$ linearly independent?
            \begin{equation*}
                \sum_{i = 1}^{n}c_iy_i = 0 \iff c_i = 0 \;\forall i \in N
            \end{equation*}
            How do you check? The Wronskian Technique:
            \begin{equation*}
                \sum c_iy_i = 0 ~;~ \sum c_iy'_i = 0 ~;~ \sum c_iy''_i = 0
            \end{equation*}
            Can be written in matrix form to solve:
            \begin{align*}
                \begin{pmatrix}
                    y_1 & y_2 & \cdots & y_n \\
                    y'_1 & y'_2 & \cdots & y'_n \\
                    \vdots & & & \\
                    v^{(n - 1)}_{1} & \cdots & \cdots & y^{n - 1}_{n}
                \end{pmatrix}
                \begin{pmatrix}
                    c_1 \\
                    c_2 \\
                    \vdots \\
                    c_n
                \end{pmatrix} = \vec{0} \\ \vec{W} \cdot \vec{C} = \vec{0}
            \end{align*}
            If $\vec{W}$ is invertible:
            \begin{align*}
                \vec{c} = (W^{-1})\cdot \vec{0} = 0 \\
                \text{det}W = |W| \neq 0
            \end{align*}
            This leads to linearly independent
            \item Solve full equation
            \item Find any solution of the full equation, the particular solution
            \item The most general solution is
                \begin{equation*}
                    y + y_p + y_c
                \end{equation*}
                if $y_p$ and $y_c$ are linearly independent
    \end{enumerate}
    \item
    \begin{align*}
        \sum_{i = 1}^{n} a_i y^(i) = 0, ~ a_i \in \R
    \end{align*}
    Try $y = Ae^{\lambda x}$:
    \begin{align*}
        y' &= \lambda y \to y'' \lambda^2 y \cdots \\
        \sum_{i = 1}^n a_i \lambda^i y &= 0 \to \sum_{i = 1}^n a_i \lambda^i = 0
    \end{align*}
    This is the auxiliary equation.
    \item $\{\lambda_i\}_{i = 1\cdots n}$ roots
    \item If all roots $\neq$: There are n solutions using equation above
    \item If some roots repeat: $\{\lambda_1,\lambda_1, \cdots\}$ \\
    This is two-fold degenerate
    \item $e^{\lambda x},\;xe^{\lambda_nx} \to$ k-fold degree
        \begin{equation*}
            \{e^{\lambda ix},xe^{\lambda ix},x^2e^{\lambda ix},\cdots,x^{k-1}e^{\lambda ix}\}
        \end{equation*}
\end{itemize}



















































\end{document}
