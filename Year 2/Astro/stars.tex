\documentclass[a4paper,11pt,normalem]{article}
\usepackage{../../../LaTeX-Templates/Notes}

\titlecontents{section}
    [0pt]
    {}
    {Lecture \thecontentslabel\quad}
    {}
    {\dotfill\contentspage}
\titleformat{\section}{\fontsize{12}{15}\normalfont}{\textbf{Lecture \thesection}}{1em}{}

\title{Stars}
\author{Prof. David Alexander}
\date{Michaelmas 2017 - Epiphany 2018}
\rhead{\hyperlink{page.1}{Contents}}

\begin{document}
\maketitle
\tableofcontents
\section{}
\emph{see DUO for slides}
Black body emission curve:
    \begin{itemize}
        \item LHS from peak lambda is Rayleigh Jeans tail
        \item RHS from peak is Wien tail
    \end{itemize}

\begin{align}
    \lambda_{max} &= \frac{2.9 \times 10^{-3}}{T} \, m \\
    \lambda_{max.\,Bet} &= 8.3 \times 10^{-7}m \implies T \approx 3500\,K \\
    \lambda_{max,Sun} &= 5.5 \times 10^{-7}m \implies T \approx 5300\,K \\
    \lambda_{max, Bel} &= 3.0 \times 10^{-7} m \implies T \approx 9400\,K\\
\end{align}

\section{}

\subsection{Excitation Energies}
\begin{itemize}
    \item Bohr model
    \item page 8 on slides
    \item n denotes the orbitals/electron shells
    \item $n = 1$ is the ground state
\end{itemize}

\begin{align}
    E &= E_{high} - E_{low} = \frac{hc}{\lambda} = -13.6\Big(\frac{1}{n_{high}^2} - \frac{1}{n_{low}^2} \Big) \\
    n &= 2 \to 4 \\
    E &= 2.55\,eV \implies \lambda = 486.1\,nm \implies H\beta
\end{align}

\begin{itemize}
    \item this was absorption
    \item \(H\beta\) is shorthand for Balmer series \(\beta\)
    \item Optical light
        \begin{align}
            n &= 2 \to 1 \\
            E &= 10.2\,eV \implies \lambda = 121.6\,nm \implies Ly\alpha
        \end{align}
    \item this was emission
    \item\(Ly\alpha\) is shorthand for Lyman series \(\alpha\)
          \begin{itemize}
            \item UV light
          \end{itemize}
    \item Photons emitted from de-excitation in random direction
        \begin{itemize}
            \item statistics means we probably won't see this
        \end{itemize}
\end{itemize}

\section{Ratios of Excitation Levels}
\begin{align}
    n &= 2 \to 1 \\
    \frac{N_2}{N_1} &= \frac{g_2}{g_1} e^{-\frac{(E_2 - E_1)}{kT}} \\
    g_1 &= 2 ~;~ g_2 = 8 ~;~ T = 5800\,K \\
    \frac{N_2}{N_1} &= 5.1 \times 10^{-9}
\end{align}
\begin{itemize}
    \item 1 billionth of H atoms in first excited state, negligible
\end{itemize}

\subsection{Ionisation Energies}
\begin{itemize}
    \item \(\chi\) is the ionisation energy
\end{itemize}
\begin{align}
    \frac{N_{i + 1}}{N_i} &= \frac{2Z_{i + 1}}{n_e Z_i} \Big(\frac{2\pi m_e kT}{h^2}\Big)^{\frac{3}{2}}e^{-\frac{\chi}{kT}} 
    E &> -13.6\Big(\frac{1}{\infty^2} - \frac{1}{n_{low}^2} \Big)\, eV \\
    n &= 1 \to \infty \implies E > 13.6\,eV \\
    n &= 2 \to \infty \implies E > 3.4\,eV
\end{align}

\section{}
\subsection{Binary Star Systems}
\begin{itemize}
    \item slide 8, binary system
    \item look at the semi-major axes of the orbits of the two stars around the
  centre of mass of the system
        \begin{itemize}
            \item \(a_1\) and \(a_2\) for \(m_1\) and \(m_2\)
        \end{itemize}
\end{itemize}
\begin{align}
    P^2 &= \frac{4\pi^2 a^3}{G(m_1 + m_2)} \\
    a &= a_1 + a_2
\end{align}

\begin{itemize}
    \item Smaller semi-major axis means larger mass
    \item similar to see-saw
\end{itemize}
\begin{align}
    m_1 a_1 = m_2 a_2 \implies \frac{m_1}{m_2} = \frac{a_2}{a_1}
\end{align}

\begin{itemize}
    \item ratio of the semi-major axes gives ratio of masses
    \item actually measure \(\alpha\), angle of separation:
        \begin{itemize}
            \item for d, distance from us
        \end{itemize}
\end{itemize}
\begin{align}
    \alpha_n = \frac{a_n}{d} \implies \frac{m_1}{m_2} = \frac{\alpha_2}{\alpha_1}
\end{align}

\subsection{Visual Binary Systems}
\paragraph{Normal Example}
\begin{itemize}
    \item \(d = 10\,pc ~;~ P = 200\) days
    \item \(\alpha_1 = 0.02" ~;~ \alpha_2 = 0.08"\)
\end{itemize}

\begin{align}
    a_1 &= \alpha_1 d = 0.2\,Au ~;~ a_2 = a_2 = \alpha_2 d = 0.8\,Au \\
    a &= a_1 + a_2 = 1\,Au \\
    m_1 + m_2 &= \frac{4\pi^2 a^3}{GP^2} = 3.4 M_\odot = M_{tot} \\
    \frac{m_1}{m_2} &= \frac{\alpha_2}{\alpha_1} = \frac{a_2}{a_1} = 4.0 = M_{rot}\\
    m_1 &= \Big[\frac{M_{rot}}{1 + M_{rot}}\Big]M_{tot} = 2.72 M_\odot\\
    m_2 &= \Big[\frac{1}{1 + M_{rot}}\Big]M_{tot} = 0.68 M_\odot
\end{align}

\paragraph{Inclination Example}
\begin{itemize}
    \item For angled systems that aren't flat against our observations:
\end{itemize}
\begin{align}
    \hat{\alpha}_n &= \alpha_n \cos i\\
    m_1 + m_2 &= \frac{4\pi^2}{G} \Big(\frac{d}{\cos i}\Big)\frac{\hat{\alpha}^3}{P^2}\\
    \hat{\alpha} &= \hat{\alpha}_1 + \hat{\alpha}_2
\end{align}
\begin{itemize}
    \item Has no effect on mass ratios observed - \(\cos\) cancels
    \item Above equation means the actual masses will be affected by the inclination
\end{itemize}

\subsection{Spectroscopic Binaries}
\begin{itemize}
\item Correcting for inclination:
\end{itemize}
\begin{align}
    v_{nr}^{max} = v_n \sin i
\end{align}
\begin{itemize}
    \item Assume \(e << 1\)
\end{itemize}
\begin{align}
    v_n &= \frac{2\pi a_n}{P}
    \frac{m_1}{m_2} &= \frac{v_2}{v_1}
\end{align}
\begin{itemize}
    \item Same sort of stuff as visual binaries, but \(\sin\) instead of \(\cos\) basically
\end{itemize}

\paragraph{Special Case: Eclipsing Spectroscopic Binaries}
\begin{itemize}
    \item \(i \approx 90^\circ\)
    \item don't need any corrections etc
\end{itemize}

\section{}
\begin{align}
    P = \underbrace{\frac{\rho kT}{\mu m_H}}_{\text{ideal gas law}} + \frac{1}{3}aT^4
\end{align}
\begin{itemize}
    \item Hydrostatic Equilibrium:
        \begin{itemize}
            \item Pressure force = Gravitational force
        \end{itemize}
\end{itemize}
\begin{align}
    P\,on\,dA &= [P(r + dr) - P(r)]dA \\
    &= dP\,dA \\
    Gravitational &= g\,\underbrace{\underbrace{dA\, dr}_{volume} \rho}_{mass},~ g = \frac{GM_r}{r^2} \\
    dP\,dA &= -g\rho\, dA\, dr \\
    \frac{dP}{dr} &= -\frac{GM_r \rho}{r^2} \\
    \frac{dM_r}{dr} &= 4\pi r^2 \rho \\
    M_r &= \frac{4}{3}\pi r^2 \rho \\
    \frac{dP}{dr} &= -G \frac{4}{3}\pi r \rho^2 \\
    \int_{P_s}^{P_c} dP &= -\frac{4}{3}\pi G \rho^2 \int_{R}^{0} r\,dr \\
    P_c &= \frac{2}{3}\pi G \rho^2 r^2,~ P_s = 0 \,at\, r = R\\
    &= \frac{2}{3}\pi Gr^2 \Big[\frac{3}{4}\frac{M}{\pi r^3}\Big]^2 \\
    &= \frac{3}{8\pi}\frac{GM^2}{R^4}
\end{align}
\begin{itemize}
    \item Example for our sun:
\end{itemize}
\begin{align}
    M &= 2\times10^{30} kg ~;~ R \approx 7\times10^8 m \\
    P_c &\approx 10^{14} N\,m^{-2} \\
    P_{c,\, true} &\approx 2\times10^{16} N\,m^{-2}
\end{align}

\begin{itemize}
    \item out as assumed uniform density
\end{itemize}

\section{}
\subsection{Virial Theorem}

\[
    \begin{aligned}
        \frac{dP}{dr} &= -\frac{GM\rho}{r^2} \times V = \frac{4}{3}\pi r^3 \\
        V\frac{dP}{dr} &= -\frac{GM\rho}{r^2}\frac{4}{3}\pi r^3 \\
        &- \text{plug in }\frac{dm}{dr} = 4\pi r^2 \rho \\
        V\frac{dP}{dr} &= \frac{1}{3}\frac{GM}{r}\frac{dm}{dr} \\
        \int_{0}^{P(R)} V\,dP &= -\frac{1}{3} \underbrace{\int_{0}^{M} \frac{GM}{r}dm}_{\text{Total GPE} = U} \\
        LHS: \int U\,dV &= UV - \int V\,dU \\
        \int_{0}^{P(R)} V\,dP &= \underbrace{[PV]_{0}^{R_{0}}}_{= 0} - \int_{0}^{V(R)} P\,dV = -\frac{1}{3} U \\
        -3 \int_{0}^{V(R)} P\,dV &= U, ~ dV = \frac{dm}{\rho} \implies \\
        -3 \int_{0}^{M} \frac{P}{\rho} dm &= U ~ \text{ - generalised form of Virial Theorem} \\
        \text{Ideal Gas: } P &= nkT = \frac{\rho kT}{\mu m_{H}} \\
        \text{Average KE: } &= \frac{3}{2}kT \\
        \text{KE per kilo: } &= \frac{3}{2} \frac{kT}{\mu m_{H}} \\
        E_{KE} &= \frac{3}{2} \frac{kT}{\mu m_{H}} = \frac{3}{2}\frac{P}{\rho} \\
        -3 \int_{0}^{M}\frac{P}{\rho}dm &= U,~ \frac{P}{\rho} = \frac{2}{3}E_{KE} \\
        \underbrace{\int_{0}^{M} E_{KE} \,dm}_{\text{Total KE, assume ideal gas}} &= -\frac{1}{2} U \\
        \implies K &= -\frac{1}{2}U
    \end{aligned}
\]

\subsubsection{Energy from Gravitational
Collapse}\label{energy-from-gravitational-collapse}

\[
    \begin{aligned}
        dU_{g,i} &= -\frac{GM_{r}dm_{i}}{r} \text{ - GPE of point mass} \\
        \text{Consider }& \text{shells of material} \\
        dm &= 4\pi r^2 \rho dr \\
        dU_{g} &= -\frac{GM_r 4\pi r^2 \rho}{r}dr \text{ - GPE of a shell} \\
        U_g &= -4\pi G \int_{0}^{R} M_r \rho_r dr \\
        M_r &= \frac{4}{3}\pi r^3 \bar{\rho} \text{ - avg density isn't too bad here} \\
        U_g &= -\frac{16}{3}\pi^2 G\bar{\rho}^2 \int_{0}^{R} r^4 dr \\
        &= -\frac{16}{15}\pi^2 G\bar{\rho}^2 R^5 \\
        \text{Convert} &\text{ back to mass} \\
        U &= -\frac{9}{15}\frac{GM^2}{R} \text{ - GPE of the star} \\
        K &= -\frac{1}{2} U \\
        \implies E &= \frac{3}{10}\frac{GM^2}{R} \\
        E &\approx \frac{3}{10} GM^2 \Big[\frac{1}{R} - \frac{1}{R_{initial}}\Big] \\
        &= \frac{3}{10}\frac{GM^2}{R} \iff R << R_{initial}
    \end{aligned}
\]

\subsection{Lecture 6}\label{lecture-6}

\subsubsection{Binding Energies of
Fusion}\label{binding-energies-of-fusion}

\[
    \begin{aligned}
    E_b(Z,N) &= \Delta mc^2 = [Zm_p + Nm_n - m(Z, N)]c^2 \\
    E_b(4,0) &= [4m_p - m_{He,4}]c^2 = 26.731\,MeV \\
    \frac{4m_p}{m_{He,4}} &= 1.007 \implies  e = 0.7\% \\
    E_{\odot} &= (0.1\times M_\odot)\times0.007\times c^2  \\
    &= 1.3\times10^{44}J \\
    t &\approx \frac{E_\odot}{L_\odot} = 10^{10}yr
    \end{aligned}
\]

\subsubsection{Coulomb Barrier}\label{coulomb-barrier}

\begin{itemize}
\item
  looking at probability that two particles are close enough for nuclear
  force to be important
\item
  see figure on page 7 of slides
\item
  using classical physics, we get
\end{itemize}

\[
    \begin{aligned}
    E &= \frac{1}{2}mv^2 = \frac{3}{2}kT = \frac{1}{4\pi\epsilon_0}\frac{Z_1Z_2e^2}{r} \\
    T &= \frac{1}{6\pi\epsilon_0}\frac{Z_1Z_2e^2}{rk} = \underbrace{1.1\times10^{10}K}_{r = 10^{-15}m ~;~ Z_1 = Z_2 = 1}
    \end{aligned}
\]

\begin{itemize}
\item
  too high for our Sun
\item
  use deBroglie wavelength and consider quantum effects
\end{itemize}

\[
    \begin{aligned}
    \lambda &= \frac{h}{p},~ p = mv ~[m = \mu_m] \\
    E &= \frac{1}{2}mv^2 ~;~ v^2 = \frac{p^2}{m^2} \\
    E &= \frac{p^2}{2m} \\
    p^2 &= \Big(\frac{h}{\lambda}\Big)^2 \\
    E &= \frac{(\frac{h}{\lambda})^2}{2m} = \frac{h^2}{\lambda^2}\frac{1}{2m} \\
    &= \frac{1}{4\pi \epsilon_0}\frac{Z_1Z_2e^2}{\lambda} = \frac{h^2}{\lambda^2}\frac{1}{2m} \\
    \frac{1}{\lambda} &= \frac{2}{4\pi\epsilon_0}\frac{Z_1Z_2e^2 m}{h^2} \\
    \text{replace }&\frac{1}{r} \text{ with }\frac{1}{\lambda} \\
    T &= \frac{1}{12\pi^2 \epsilon_{0}^2} \frac{Z_1^2Z_2^2e^4 m}{kh^2} = 9.8\times10^6 K
    \end{aligned}
\]

\begin{itemize}
\item
  this happens due to quantum tunneling
\end{itemize}

\subsubsection{Probability of Nuclear
Reactions}\label{probability-of-nuclear-reactions}

\begin{itemize}
\item
  see graph on page 13 of slides
\item
  nuclear reaction probability is the product of Maxwell-Boltzmann and
  Tunneling Probability
\end{itemize}

\subsection{Lecture 7}\label{lecture-7}

\subsubsection{Nuclear Conservation
Rules}\label{nuclear-conservation-rules}

\begin{enumerate}
\def\labelenumi{\arabic{enumi}.}
\item
  electric charge must be conserved
\item
  nucleon umber must be conserved

  \begin{itemize}
  \item
    \(p, n = + 1\)
  \end{itemize}
\item
  lepton number must be conserved

  \begin{itemize}
  \item
    \(e^\mp = \pm 1\)
  \item
    \(\nu_{e}^\mp = \pm 1\)
  \end{itemize}
\end{enumerate}

\[
    ^{A}_{Z}X
\]

\begin{itemize}
\item
  A - atomic number for element X (nucleon number)
\item
  Z - number of protons (electric charge)
\end{itemize}

\subsubsection{Proton-Proton Chains}\label{proton-proton-chains}

\[
    \begin{aligned}
    {}^{1}_{1}H &+ {}^{1}_{1}H \to {}^{2}_{1}H + e^+ + \nu_e \\
    {}^{2}_{1}H &+ {}^{1}_{1}H \to {}^{3}_{2}He + \gamma \\
    {}^{3}_{2}He &+ {}^{3}_{2}He \to {}^{4}_{2}He + {}^{1}_{1}H + {}^{1}_{1}H \\
    \implies 4{}^{1}_{1}H &\to {}^{4}_{2}He + \underbrace{2e^+ + 2\nu_e + 2\gamma}_{26.7\,MeV}
    \end{aligned}
\]

\subsubsection{CNO Cycle}\label{cno-cycle}

\[
    \begin{aligned}
    {}^{12}_{6}C &+ {}^{1}_{1}H \to {}^{13}_{7}N + \gamma \\
    {}^{13}_{7}N  &\to \underbrace{{}^{13}_{6}C + e^+ + \nu_e}_{\beta\text{ decay}} \\
    {}^{13}_{6}C &+ {}^{1}_{1}H \to {}^{14}_{7}N + \gamma \\
    {}^{14}_{7}N &+ {}^{1}_{1}H \to {}^{15}_{8}O + \gamma \\
    {}^{15}_{8}O &\to \underbrace{{}^{15}_{7}N + e^+ \nu_e}_{\beta\text{ decay}} \\
    {}^{15}_{7}N &+ {}^{1}_{1}H \to {}^{12}_{6}C + {}^{4}_{2}He \\
    \text{Total: }4{}^{1}_{1}H &\to {}^{4}_{2}He + \underbrace{2e^+ + 2\nu_e + 3\gamma}_{E = 26.7\,MeV}
    \end{aligned}
\]

\subsection{Lecture 8}\label{lecture-8}

\subsubsection{Energy produced in Stars}\label{energy-produced-in-stars}

\[
    \begin{aligned}
    dL &= \epsilon\,dm ~~ [W] \\
    \epsilon_{i,X} &= \epsilon_0 X_i X_X \rho^\alpha T^\beta ~~ [W\,kg^{-1}] \\
    dm &= 4\pi r^2\rho\,dr \\
    \implies \frac{dL}{dr} &= 4\pi r^2 \rho \epsilon
    \end{aligned}
\]

\paragraph{Slide 5 diagram}\label{slide-5-diagram}

\begin{itemize}
\item
  Solid line just to do with fusion then no fusion
\item
  Dashed line has that shape as volume increase so dL/dr does but then
  temperature starts falling so fusion decreases
\end{itemize}

\subsubsection{Energy Seen on Earth}\label{energy-seen-on-earth}

\begin{itemize}
\item
  Electrons lose energy travelling through sun
\end{itemize}

\[
    \frac{\lambda_{surface}}{\lambda_{core}} \approx 3\times10^6
\]

\subsubsection{Mean Free Paths}\label{mean-free-paths}

\begin{itemize}
\item
  \(vt\) - distance travelled
\item
  \(n\) - particles per unit volume
\item
  \(nvt\) - particle per unit area
\item
  \(n\sigma vt\) - number of interactions
\end{itemize}

\[
    \begin{aligned}
    l &= \frac{vt}{n\sigma vt} \\
    &= \frac{1}{n\sigma}
    \end{aligned}
\]

\begin{itemize}
\item
  This is the mean distance before a collision
\end{itemize}

\[
    \begin{aligned}
    d &= \sum_i l_i \\
    d^2 &= d \cdot d \\
    &= \sum_j \sum_i l_i \cdot l_j
    \end{aligned}
\]

\begin{itemize}
\item
  When \(i \neq j\), \(l_i \cdot l_j = 0\)
\end{itemize}

\[
    \begin{aligned}
    d^2 &= Nl^2 \\
    \implies N = \bigg(\frac{d}{l}\bigg)^2
    \end{aligned}
\]

\begin{itemize}
\item
  Use ideal gas law to help in questions of this
\end{itemize}

\[
    \begin{aligned}
    t_{total} &= t_{travel} + Nt_{scatter} \\
    &= \frac{Nl}{c} + N\times 10^8 \\
    &= 5700\;yrs + \cdots = 10^6 \, yrs
    \end{aligned}
\]

\subsubsection{Radiation}\label{radiation}

\[
    \begin{aligned}
    P &= \frac{1}{3}a T^4 \\
    \frac{dP}P{dr} &= \frac{dP}{dT}\frac{dT}{dr} \\
    \frac{dP}{dr} &= \frac{4}{3}a T^3 \frac{dT}{dr} \\
    \frac{dP}{dr} &= -\frac{\kappa\rho}{c}F_{rad} \\
    \kappa rho &= n\sigma \\
    \frac{dT}{dr} &= -\frac{3}{4ac}\frac{\kappa\rho F_{rad}}{T^3} \\
    L &= 4\pi r^2 F_{rad} \\
    \frac{dT}{dr} &= -\frac{3}{16\pi ac}\frac{\kappa\rho L_r}{T^3 r^2}
    \end{aligned}
\]

\subsection{Lecture 9}\label{lecture-9}

\subsubsection{Opacity}\label{opacity}

\[
    \begin{aligned}
    dI_\lambda &= - \kappa_\lambda \rho I_{\lambda} ds \\
    \int_{I_{\lambda, 0}}^{I_{\lambda, f}} \frac{dI_{\lambda}}{I_{\lambda}} &= - \int \kappa_{\lambda} \rho ds \\
    \implies I_{\lambda, f} &= I_{\lambda, 0}e^{-\int_0^s \kappa_{\lambda}\rho ds} \\
    I_{\lambda, f} &= I_{\lambda, 0}\underbrace{e^{-\kappa_\lambda \rho s}}_{\text{optical depth, }\tau} \\
    &= I_{\lambda, 0}e^{-\tau}, ~ \tau = \kappa_{\lambda}\rho s
    \end{aligned}
\]

\begin{itemize}
\item
  \(\tau < 1\) - optically thin
\item
  \(\tau > 1\) - optically thick
\end{itemize}

\paragraph{Different sources of
Opacity}\label{different-sources-of-opacity}

\begin{itemize}
\item
  Two classes of opacity:

  \begin{enumerate}
  \def\labelenumi{\arabic{enumi}.}
  \item
    Absorption - photon energy lost of KE of gas or degraded
  \item
    Scattering - photon reemitted at different direction, sometimes
    degraded
  \end{enumerate}
\end{itemize}

\begin{enumerate}
\def\labelenumi{\arabic{enumi}.}
\item
  Bound-Bound transitions

  \begin{itemize}
  \item
    typical temperature roughly \(\leq 10^5\)K
  \item
    most effective for neutral gas
  \item
    scattering and absorption
  \end{itemize}
\item
  Bound-free transitions

  \begin{itemize}
  \item
    typical temperature of \(10^4 \to 10^6\)K
  \item
    partially ionised gas
  \item
    absorption
  \end{itemize}
\item
  Free-free emission

  \begin{itemize}
  \item
    typical temperature of \(10^4 \to 10^6\)K
  \item
    partially ionised gas
  \item
    absorption
  \end{itemize}
\item
  Electron scattering

  \begin{itemize}
  \item
    dominant at roughly \(\geq 10^6\)K
  \item
    fully ionised gas
  \item
    scattering
  \end{itemize}
\end{enumerate}

\subsection{Lecture 10}\label{lecture-10}

\subsubsection{Schwarzchild Criterion for
Convection}\label{schwarzchild-criterion-for-convection}

\begin{itemize}
\item
  slide 4 - 9
\end{itemize}

\[
    \gamma = \frac{C_p}{C_V} = \frac{s + 2}{s}
\]

\begin{itemize}
\item
  s is degrees of freedom
\end{itemize}

\[
    \begin{aligned}
    P &= k_a \rho^\gamma \\
    \frac{dP}{P} &= \frac{\gamma d\rho}{\rho} \\
    \gamma &= \frac{\rho}{P}\frac{dP}{d\rho} \\
    \text{Surrou}&\text{nding gas} \\
    P &= nkT = \frac{\rho kT}{\mu m_H} \\
    \frac{dP}{P} &= \frac{d\rho}{\rho} + \frac{dT}{T} \\
    \frac{d\rho}{\rho} &= \frac{dP}{P} - \frac{dT}{T} \\
    \frac{dP}{d\rho}_{sur} &> \frac{dP}{d\rho}_{adiab} \Bigg[\times \frac{\rho}{P} \\
    \frac{\rho}{P}\frac{dP}{d\rho}_{sur} &> \frac{\rho}{P} \frac{dP}{d\rho}_{adiab} \\
    \frac{\rho}{P}\frac{dP}{d\rho}_{sur} &> \gamma_{ad} \\
    \frac{P}{dP}\Big(\frac{dP}{P} &- \frac{dT}{T}\Big)_{sur} < \frac{1}{\gamma_{adiab}} \\
    \frac{P}{dP}\frac{dP}{P} &- \frac{P}{dP}\frac{dT}{T} < \frac{1}{\gamma_{adiab}} \\
    1 - \Big(\frac{P}{dP}&\frac{dT}{T}\Big)_{sur} < \frac{1}{\gamma_{adiab}} \\
    \frac{T}{P} \Big(\frac{dP}{dT}\Big)_{sur} &< \frac{\gamma_{adiab}}{\gamma_{adiab} - 1} \\
    \Big|\frac{dT}{dr}\Big|_{sur} &> \Big(\frac{\gamma_{adiab} - 1}{\gamma_{adiab}}\Big)\frac{T}{P} \Big|\frac{dP}{dr}\Big|_{sur_{}}
    \end{aligned}
\]

\paragraph{Convection in the Sun}\label{convection-in-the-sun}

For the sun:

\[
    \begin{aligned}
    -\frac{3}{16\pi a c}\frac{k\rho L_r}{T^3 r^2} &>  \Big(\frac{\gamma_{} - 1}{\gamma_{}}\Big)\frac{T}{P} \frac{dP}{dr} \\
    \frac{dP}{dr} &= -\frac{GM_r \rho}{r^2} \\
    \frac{L_r}{M_r} &> \frac{16 \pi a c G}{\kappa\rho} \frac{aT^4}{3} \frac{\gamma - 1}{\gamma} \\
    &> \frac{16 \pi acG}{\kappa\rho}P_{rad}\frac{\gamma - 1}{\gamma} \\
    &> 1.9\times10^{-3}\,W\,kg^{-1}
    \end{aligned}
\]

\paragraph{Mixing length}\label{mixing-length}

\[
    \begin{aligned}
    l &= \alpha Hp \\
    \frac{dP}{dr} = - \frac{GM_r \rho}{r^2} &\implies \frac{1}{Hp} = -\frac{1}{P}\frac{dP}{dr} \\
    Hp &= \frac{Pr^2}{GM_r \rho} \\
    l&= \frac{\alpha Pr^2}{GM_r \rho}
    \end{aligned}
\]

\subsection{Lecture 12}\label{lecture-12}

\subsubsection{Cepheid Variables}\label{cepheid-variables}

\[
    \begin{aligned}
    \log\bigg(\frac{L}{L_{\odot}}\bigg) &= 1.15\log_{10}\Pi^d + 2.47 \\
    \Pi^d = 10\,\text{days} &\implies L = 4200\,L_{\odot} \\
    \text{observed} <f> &= 10^{-15} W\,m^{-2} \\
    L &= 4\pi d^2 <f> \\
    d &= \sqrt{\frac{L}{4\pi<f>}}
    \end{aligned}
\]

\subsubsection{Stellar Pulsation}\label{stellar-pulsation}

\[
    \begin{aligned}
    V_s &= \sqrt{\frac{\gamma P}{\rho}}, ~ \gamma = \frac{C_p}{C_V} \\
    \Pi &= 2\int_0^R \frac{dr}{V_s} \\
    \frac{dP}{dr} &= -\frac{GM_r \rho}{r^2} \\
    \text{const } p &\implies \mu = \frac{4}{3}\pi r^3 \rho \\
    \frac{dP}{dr} &= -\frac{4}{3}G\pi r \rho^2 \\
    dP &= -\frac{4}{3}G\pi\rho^2 \int_0^R r\,dr \\
    P(r) &= \frac{4}{3}G\pi\rho^2 \bigg[\frac{R^2}{2} - \frac{r^2}{2}\bigg] \\
    \Pi &= 2 \int_0^R \frac{dr}{V_s} \\
    &= 2\int_0^R \frac{dr}{\sqrt{\frac{2}{3}\gamma G\rho (R^2 - r^2)}} \\
    &= 2\sqrt{\frac{3}{2\gamma\pi G \rho}} \Bigg[\sin^{-1} \bigg(\frac{r}{R}\bigg)\Bigg]^{R}_{0} \\
    &= \sqrt{\frac{3\pi}{2G\rho\gamma}}
    \end{aligned}
\]

\subsection{Lecture 13}\label{lecture-13}

\subsubsection{Jeans Mass}\label{jeans-mass}

\begin{itemize}
\item
  For the gravitational collapse of a gas cloud:
\end{itemize}

\[
    \begin{aligned}
    GE = U &= -\frac{3}{5}\frac{GM^2}{R} \\
    KE = K &= \frac{3}{2}NkT \\
    &= \frac{3}{2}\frac{M_c}{\mu m_H}kT \\
    2K &< |U| \\
    2\Bigg(\frac{3}{2}\frac{M_c kT}{\mu m_H}\Bigg) &< \frac{3}{5}\frac{GM_c^2}{R_c} \\
    R_c &= \Bigg(\frac{3}{4}\frac{M_c}{\pi \rho_0}\Bigg)^{\frac{1}{3}} \\
    2\Bigg(\frac{3}{2}\frac{M_c kT}{\mu m_H}\Bigg) &< \frac{3}{5}GM_c^2 \Bigg(\frac{4}{3}\frac{\pi \rho_0}{M_c}\Bigg)^{\frac{1}{3}} \\
    \frac{5M_c kT}{\mu mH G} &< M_c^2 \Bigg(\frac{4}{3}\frac{\pi \rho_0}{M_c}\Bigg)^{\frac{1}{3}} \\
    M_c &< M_J \\
    M_J &\approx \Bigg(\frac{5kT}{G\mu m_H}\Bigg)^{\frac{3}{2}} \Bigg(\frac{3}{4\pi\rho_0}\Bigg)^{\frac{1}{2}}
    \end{aligned}
\]

\subsubsection{Free-fall gravitational
collapse}\label{free-fall-gravitational-collapse}

\begin{enumerate}
\def\labelenumi{\arabic{enumi}.}
\item
  \(M_c > M_J\)

  \begin{itemize}
  \item
    free fall collapse
  \item
    optically thin
  \item
    pressure increase
  \item
    temperature constant
  \end{itemize}
\item
  Fragmentation

  \begin{itemize}
  \item
    optically thin
  \item
    individual regions exceed local \(M_J\)
  \end{itemize}
\item
  \(M_J\) minimised: Protostar

  \begin{itemize}
  \item
    optically thick
  \item
    pressure increase
  \item
    temperature increase
  \item
    Slow contraction (Kelvin-Helmholtz timescale)
  \end{itemize}
\end{enumerate}

\subsection{Lecture 14}\label{lecture-14}

\subsubsection{Stellar Evolution}\label{stellar-evolution}

\begin{enumerate}
\def\labelenumi{\arabic{enumi}.}
\item
  Increase in \(\mu\) (mean molecular mass) with time:
\end{enumerate}

\[
    P = nkT = \frac{\rho kT}{\mu m_H}
\]

As \(\mu\) increases, \(\rho\) and T also increase for the pressure to
remain constant.

Recall:

\[
    \epsilon_{i,X} = \epsilon_0 X_i X_X \rho^\alpha T^\beta, \alpha \approx 1
\]

For proton-proton chain, \(\beta \approx 4\)\\
For CNO, \(\beta \approx 17\)

Luminosity increases with time.

\subsubsection{Lifetime of Nuclear
Fusion}\label{lifetime-of-nuclear-fusion}

\[
    \begin{aligned}
    t &= \frac{E_{tot}}{L} = \frac{X\zeta Mc^2}{L} \\
    \zeta_{pp} &= \frac{4m_p - m_{He}}{m_{He}} \approx 0.007 \\
    t_{\odot} &= 10^{10} \text{ yrs} \\
    L_{ms} &= L_\odot \left(\frac{M_\odot}{M}\right)^{\alpha} \\
    t_{ms} &= \frac{X\zeta Mc^2}{L_\odot}\left(\frac{M_\odot}{M}\right)^{\alpha} \\
    &= 10^{10}\frac{M}{M_\odot}\left(\frac{M_\odot}{M}\right)^{\alpha} \\
    \therefore t_{ms} &= 10^{10}\left(\frac{M_\odot}{M}\right)^{\alpha - 1}
    \end{aligned}
\]

\subsection{Lecture 15}\label{lecture-15}

\subsubsection{Eddington Limit}\label{eddington-limit}

\[
    \begin{aligned}
    L_{Edd} &= \frac{4\pi cGM}{\kappa}, M = 100M_\odot, \kappa = \kappa_{es} = 0.04\,kg\,m^{-2} \\
    &= 3\times10^6 L_{\odot}
    \end{aligned}
\]

\subsubsection{Photodisintegration}\label{photodisintegration}

\[
    \begin{aligned}
    \lambda_{max} &= \frac{2.9\times10^{-3}}{T}, ~ E = \frac{hc}{\lambda} \\
    T_c &\geq 3\times10^9 K \implies E \geq 1\,MeV
    \end{aligned}
\]

\paragraph{Last Days of Fusion}\label{last-days-of-fusion}

\begin{itemize}
\item
  Shell fusion
\item
  Silicon to Iron in Core
\item
  \(P_{core} =\) high
\end{itemize}

\paragraph{Endothermic Release}\label{endothermic-release}

\begin{itemize}
\item
  Iron breaking down into Helium and Helium breaking down in protons and
  neutrons
\item
  still shell fusion ongoing
\item
  \(P_{core} =\) rapidly decreasing
\end{itemize}

\paragraph{Electron capture}\label{electron-capture}

\begin{itemize}
\item
  very high density
\item
  shell fusion
\item
  \(p + e^- \implies n + \nu_e\)
\item
  \(P_{core} =\) rapidly decreasing
\item
  neutrino burst
\end{itemize}

\paragraph{Rapid core collapse}\label{rapid-core-collapse}

\begin{itemize}
\item
  shell fusion
\item
  \(P_{core} \approx 0\)
\end{itemize}

\paragraph{Core rebound}\label{core-rebound}

\begin{itemize}
\item
  shell fusion
\item
  \(\rho > 8\times10^{18}\,kg\,m^{-3}\)
\item
  the strong force repels collapse and rebounds outwards
\end{itemize}

\paragraph{Supernova}\label{supernova}

\begin{itemize}
\item
  previous step drives supernova
\item
  strong force drives high energy pushing
\item
  generates a shock wave - more photodisintegration
\item
  electron capture repeats and another neutrino burst
\item
  nuclear synthesis of heavier elements, including beyond iron
  (endothermic)
\end{itemize}

\subsection{Lecture 16}\label{lecture-16}

\subsubsection{Electron Degeneracy
Pressure}\label{electron-degeneracy-pressure}

\[
    \begin{aligned}
    \Delta x\Delta p_x &\approx \hbar \\
    p_min &\approx \Delta p_x \approx \frac{\hbar}{\Delta x} \\
    P &\approx \frac{1}{2}n_e pv \\
    n_e &= \frac{\# e}{vol} = \frac{Z}{A} \frac{\rho}{m_H} \\
    p_x &= \Delta p_x = \frac{\hbar}{\Delta x} \\
    \Delta x &= n_e^{-1/3} \implies p_x = \hbar n_e^{1/3} \\
    p^2 &= p_x^2 + p_y^2 + p_z^2 = 3p_x^2 \\
    \implies p &= \sqrt{3}p_x = \sqrt{3}\hbar n_e^{1/3} \\
    p &= mv = m_e v \\
    \implies v &= \frac{p}{m_e} = \frac{\sqrt{3}}{m_e}\hbar n_e^{1/3} \\
    P &= \frac{1}{3}n_e pv \\
    p &= \sqrt{3}\hbar \left[\left(\frac{Z}{A}\right)\frac{\rho}{m_H}\right]^{1/3} \\
    v &= \frac{\sqrt{3}}{m_e}\hbar \left[\left(\frac{Z}{A}\right) \frac{\rho}{m_H}\right]^{1/3} \\
    \therefore P &= \frac{\hbar^2}{m_e}\left[\left(\frac{Z}{A}\right) \frac{\rho}{m_H}\right]^{5/3}
    \end{aligned}
\]

\subsubsection{White Dwarf Cooling}\label{white-dwarf-cooling}

\[
    \begin{aligned}
    t_{cool} = \frac{E_{WD}}{L_{WD}} = \left(\frac{3kT_{c,WD}}{2}\right) \left(\frac{M_{WD}}{Am_H}\right) \left(\frac{1}{L_{WD}}\right)
    \end{aligned}
\]

\subsection{Lecture 17}\label{lecture-17}

\subsubsection{Rotation Period of
Pulsars}\label{rotation-period-of-pulsars}

\[
    \begin{aligned}
    \text{Centripetal Acceleration} &= \text{Gravitational Acceleration} \\
    \omega^2_{max}R &= \frac{GM}{R} \\
    M &= \frac{4}{3}\pi R^3\rho \\
    \omega_{max}^2 R &= G \frac{4}{3}\pi R \rho \\
    \omega &= 2\pi f = \frac{2\pi}{P} \\
    \frac{4\pi^2}{P^2}R &= \frac{4}{3}G\pi R\rho \\
    P_{min} &= \left(\frac{3\pi}{G\rho}\right)^{1/2}
    \end{aligned}
\]

\subsubsection{Stellar Core Rotation}\label{stellar-core-rotation}

Conservation of angular momentum:

\[
    \begin{aligned}
    I_i\omega_i &= I_f\omega_f, ~ I = CMR^2 \\
    CMR_i^2\omega_i &= CMR_f^2\omega_f, ~ \omega = \frac{2\pi}{P} \\
    \frac{2\pi}{P_f} &= \frac{2\pi}{P_i}\left(\frac{R_i}{R_f}\right)^2 \\
    P_f &= P_i\left(\frac{R_f}{R_i}\right)^2
    \end{aligned}
\]
\end{document}
