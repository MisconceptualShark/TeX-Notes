\documentclass[a4paper, 11pt, normalem]{report}

\usepackage{../../../LaTeX-Templates/Notes}
\usepackage{subfiles}

\title{Particle Theory \vspace{-20pt}}
\author{Prof Krauss, Prof Khoze, and Dr Alonso}
\date{\vspace{-15pt}Michaelmas Term 2019 - Epiphany Term 2020}
\rhead{\hyperlink{page.1}{Contents}}

\begin{document}

\maketitle
\tableofcontents

\part{Relativistic Quantum Mechanics}
\chapter{Recapitulation of important ingredients}

\section{Natural Units}
\begin{align}
    \hbar &= c = 1 \\
    \text{energy} &= \frac{1}{\text{length}} \\
    \hbar c &= 200\,\text{MeV}\cdot\text{fm} \\
    \text{energy} &= \text{momentum}~ (c = 3\times10^8\;\text{ms}^{-1} = 3\times10^{23}\;\text{fm s}^{-1})
\end{align}

\section{Four Vectors}
\begin{align}
    p^{\mu=\{0,\dots,3\}} &= (E,\unl{p}) \\
    p^\mu p_\mu &= E^2 - \unl{p}^2 = E^2 - p_ip_i \\
    g^{\mu\nu} &= \begin{pmatrix} 1 & 0 & 0 & 0 \\ 0 & -1 & 0 & 0 \\ 0 & 0 & -1 & 0 \\ 0 & 0 & 0 & -1 \end{pmatrix} = g_{\mu\nu} \\
    g^\mu_\nu &= I \\
    \frac{\p}{\p x^\mu} &= \p_\mu \\
    \frac{\p}{\p x_\mu} &= \p^\mu \\
    \frac{\p}{\p x^\mu} (x\cdot p) &= p_\mu \\
    \frac{\p}{\p x_\mu} (x\cdot p) &= p^\mu
\end{align}

Kronecker delta:
\begin{align}
    \delta_{ij} = \delta^{ij} &= \begin{cases} 1 & i=j \\ 0 & \text{otherwise} \end{cases} 
\end{align}
Levi-Civita tensor:
\begin{equation}
    \e_{ijk} = \e^{ijk} = \begin{cases} 1 & \{ijk\} \text{ cyclical perm of 123} \\ -1 & \{ijk\} \text{ anti-cyclical perm} \\ 0 & \text{otherwise} \end{cases}
\end{equation}
Anti-symmetric tensor:
\begin{equation}
    \e_{\mu\nu\rho\sigma} = \e^{\mu\nu\rho\sigma} = \begin{cases} 1 & \{\mu\nu\rho\sigma\} \text{ cyclical perm of 0123} \\ -1 & \{\mu\nu\rho\sigma\} \text{ anti-cyclical perm} \\ 0 & \text{otherwise} \end{cases}
\end{equation}

\section{Lorentz Transformation}
Boosts and rotations:
\begin{equation}
    x'^\mu = \Lambda^\mu_\nu x^\nu
\end{equation}
Lorentz boost along z-axis:
\begin{equation}
    \Lambda^\mu_\nu = \begin{pmatrix} \cosh\nu & 0 & 0 & -\sinh\nu \\ 0 & 1 & 0 & 0 \\ 0 & 0 & 1 & 0 \\ -\sinh\nu & 0 & 0 & \cosh\nu \end{pmatrix}
\end{equation}
where rapidity
\begin{equation}
    \cosh\nu = \gamma - \frac{1}{\sqrt{1-\nu^2}}
\end{equation}

\section{Lagrange formalism}
\begin{align}
    L\left(q(t),\dot{q}(t)\right) &= T - V \\
    S(t,t_0) &= \int_{t_0}^t L\; dt' 
\end{align}
Minimise action, $S$, leads to Euler-Lagrange equations of motion:
\begin{align}
    \frac{d}{dt} \frac{\p L}{\p\dot{q}} - \frac{\p L}{\p q} = 0
\end{align}
Introduce Hamilton function:
\begin{align}
    H(p,q) &= \dot{q}\frac{\p L}{\p \dot{q}} - L = T+V,~ \dot{q} \to p = \frac{\p L}{\p \dot{q}} \\
    \dot{q} &= \frac{\p H}{\p p},~ \dot{p} = -\frac{\p H}{\p q}
\end{align}

\section{Harmonic Oscillator, 1st quantisation}
\begin{align}
    \hat{H} &= \frac{\hat{p}^2}{2m} + \frac{m\om^2}{2}\hat{x}^2 \\
    [\hat{x},\hat{p}] &= i = \hat{x}\hat{p} - \hat{p}\hat{x}
\end{align}
Introducing the annihilation and creation operators:
\begin{align}
    \hat{a} &= \frac{1}{\sqrt{2}}\left(\sqrt{\om}\hat{x} + \frac{i}{\sqrt{\om}}\hat{p}\right) \\
    \hat{a}^\dagger &= \frac{1}{\sqrt{2}}\left(\sqrt{\om}\hat{x} - \frac{i}{\sqrt{\om}}\hat{p}\right)
\end{align}
$\hat{x}$,$\hat{p}$,$\hat{H}$ are Hermitian, so
\begin{align}
    [\hat{a},\hat{a}^\dagger] &= 1 \\
    [\hat{a},\hat{a}] &= [\hat{a}^\dagger,\hat{a}^\dagger] = 0 \\
    \hat{H} &= \om\left(\hat{a}^\dagger\hat{a} + \frac12\right) = \om\left(\hat{N}+\frac12\right) \\
    \hat{N}|n\rangle &= n|n\rangle \\
    [\hat{N},\hat{a}^\dagger] &= \hat{a}^\dagger \\
    [\hat{N},\hat{a}] &= -\hat{a} \\
    \hat{H}|E\rangle &= E|E\rangle 
    \hat{H}\left(\hat{a}|E\rangle\right) &= \left(\hat{a}\hat{H} + \hat{H}\hat{a} - \hat{a}\hat{H}\right)|E\rangle \\
                                         &= aE|E\rangle + \om[\hat{N},\hat{a}]|E\rangle \\
                                         &= (E-\om)\hat{a}|E\rangle
\end{align}
Eigenvalues of Hermitian operators are real numbers, therefore the eigenvalues of their squares cannot be negative $\implies$ there must be a lowest state $|0\rangle$ (the ground state) such that
\begin{align}
    \hat{a}|0\rangle &= 0 \implies E_0 = \frac{\om}{2}
\end{align}

\end{document}



















