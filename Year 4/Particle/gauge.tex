\documentclass[relqm.tex]{subfiles}

\begin{document}
\part{Gauge Theories}
\chapter{}
\section{Plan}
\begin{itemize}
    \item Lectures 1-2: Introduction and Motivation, Intro to Group theory $\to$ Lie groups (continoues symmetries).
        \begin{itemize}
            \item Why group theory? Gauge theories are quantum field theories with an emphasis on symmetry, as gauge theories have gauge symmetries. Mathematically, symmetries are described by group theory. 
        \end{itemize}
    \item Lecture 3: Different types of symmetries - global and local (gauge) symmetries. 
    \item Lecture 4: From these gauge symmetries, we will construct Abelian, and non-Abelian gauge field theories. 
    \item By the end of the course, we will learn about the Higgs mechanism and Spontaneous Symmetry Breaking, and ultimately discuss the full Standard Model of Particle Physics.
\end{itemize}
The Standard Model is a gauge field theory of SU(3)$\times$SU(2)$\times$U(1) - this is the gauge group of the Standard Model. 
\begin{itemize}
    \item SU(3) is the gauge theory of strong interactions (QCD).
    \item SU(2)$\times$U(1) is the gauge theory of the unified electroweak interactions.
\end{itemize}

\section{Introduction to Group Theory}
Groups are needed in order to describe and define symmetry transformations. 
So what is a group?

There are four properties that define a group:
\begin{enumerate}
    \item \textbf{Closure} under group multiplication. 
        \begin{align}
            g_1\cdot g_2 = g_3 \in G,\; \forall g_1,g_2 \in G
        \end{align}
        So group multiplication is confined to within the bounds of the group.
    \item \textbf{Associativity} of group multiplication. 
        \begin{align}
            g_1\cdot(g_2\cdot g_3) &= (g_1\cdot g_2)\cdot g_3,\; \forall g_1,g_2,g_3 \in G
        \end{align}
    \item \textbf{Identity element}. 
        \begin{align}
            \exists\, e \in G\,:\, e\cdot g = g\cdot e = g,\; \forall g \in G
        \end{align}
    \item \textbf{Inverse element}.
        \begin{align}
             \exists\, g^{-1} \in G\,:\, g^{-1}\cdot g = g\cdot g^{-1} = e,\; \forall g \in G
        \end{align}
\end{enumerate}

Now some notes:
\begin{itemize}
    \item The group is called \textbf{Abelian} iff
        \begin{align}
            g_1\cdot g_2 = g_2\cdot g_1,\; g_1,g_2 \in G
        \end{align}
        This is equivalent to being called commutative, from $[g_1,g_2]=0$.
    \item Then it holds that we have \textbf{non-Abelian} groups, where
        \begin{align}
            g_1\cdot g_2 \neq g_2\cdot g_1
        \end{align}
        This is non-commutative, from $[g_1,g_2]\neq0$.
\end{itemize}
Matrix multiplication (of square matrices) is an example of a non-Abelian group multiplication.

\subsection{Important Examples}
\begin{itemize}
    \item SU(N) is a group of unitary $N\times N$ matrices, with $\text{det}=1$. For SU(N), N is for $N\times N$ matrices, and the U says it is unitary. 
        Unitary is defined by
        \begin{align}
            SU(N) \ni U\,:\, U^\dagger\cdot U = U\cdot U^\dagger = \mathbb{I}_{N\times N},~ U^\dagger = (U^*)^T
        \end{align}
        where $U^\dagger$ is called Hermitian conjugation. 
        The S then defines the group as "Special", which means $\text{det}(U) = 1$.\\
        Let's check these group properties. 
        \begin{enumerate}
            \item Matrix multiplication:
                \begin{align}
                    U_1 &\in SU(N):~ U_1^\dagger U_1 = \mathbb{I} \\
                    U_2 &\in SU(N):~ U_2^\dagger U_2 = \mathbb{I} \\
                    (U_1U_2)^\dagger U_1U_2 &= U_2^\dagger \underbrace{U^\dagger_1 U_1}_{\mathbb{I}}U_2 = U^\dagger_2 U_2 = \mathbb{I} \\
                    \text{det}(U_1U_2) &= \text{det}(U_1)\cdot\text{det}(U_2) = 1
                \end{align}
                So we have \textbf{closure.}
            \item Associativity is satisfied by the definition of matrix multiplication.
            \item Unit matrix:
                \begin{align}
                    e &= \mathbb{I}_{N\times N} 
                \end{align}
            \item The inverse matrix element:
                \begin{align}
                    U^{-1} &= U^\dagger
                \end{align}
        \end{enumerate}
        So we have a group that holds all the properties, a very important group at that. \\
        Consider some general $U \in SU(N)$. 
        How many real independent parameters (real degrees of freedom) does $U$ have?
        An $N\times N$ complex matrix will have $2N^2$ real degrees of freedom. 
        Now if we require unitarity, $U^\dagger = U$, there are $N^2$ constraints on the degrees of freedom, so now we are left with only $N^2$ degrees of freedom by this requirement. 
        Now if we impose that $\text{det}U = 1$, which is a single condition, we are left with $N^2-1$ real degrees of freedom. 
    \item U(1) is a group of unitary $1\times1$ matrices, so $U^\dagger U = 1$. 
        \begin{align}
            \forall U \in U(1)\,:\, U = e^{i\alpha},\; \alpha \in \R
        \end{align}
        Note we cannot require that the $\text{det}U=1$, otherwise we collapse down to a single value of this group, where $\alpha=0$.
        We do not really need to check the group properties of U(1) as they are completely trivial.
    \item SO(N) is a group of $N\times N$ real-valued matrices which are orthogonal:
        \begin{align}
            \forall O \in SO(N)\,:\, O^T\cdot O = O\cdot O^T = \mathbb{I}
        \end{align}
        So N is for $N\times N$, O is for orthogonal, and S again for $\text{det}=1$.\\
        SO(N) matrices are \emph{proper} (we do not do parity transformations of $x\to -x$) rotations in the $\R^N$ (N-dimensional real vector space).
        It is again trivial to find the four group properties fully satisfied for SO(N), so these are groups again.
        \begin{align}
            SO(2) \ni O = \begin{pmatrix} \cos\theta & \sin\theta \\ -\sin\theta & \cos\theta\end{pmatrix}
        \end{align}
\end{itemize}

\end{document}



















