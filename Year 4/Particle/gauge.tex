\documentclass[relqm.tex]{subfiles}

\begin{document}
\part{Gauge Theories}
\chapter{}
\section{Plan}
\begin{itemize}
    \item Lectures 1-2: Introduction and Motivation, Intro to Group theory $\to$ Lie groups (continoues symmetries).
        \begin{itemize}
            \item Why group theory? Gauge theories are quantum field theories with an emphasis on symmetry, as gauge theories have gauge symmetries. Mathematically, symmetries are described by group theory. 
        \end{itemize}
    \item Lecture 3: Different types of symmetries - global and local (gauge) symmetries. 
    \item Lecture 4: From these gauge symmetries, we will construct Abelian, and non-Abelian gauge field theories. 
    \item By the end of the course, we will learn about the Higgs mechanism and Spontaneous Symmetry Breaking, and ultimately reach the full Standard Model of Particle Physics.
\end{itemize}
The Standard Model is a gauge field theory of SU(3)$\times$SU(2)$\times$U(1) - this is the gauge group of the Standard Model. 
\begin{itemize}
    \item SU(3) is the gauge theory of strong interactions (QCD).
    \item SU(2)$\times$U(1) is the gauge theory of the unified electroweak interactions.
\end{itemize}

\section{Introduction to Group Theory}
Groups are needed in order to describe and define symmetry transformations. 
So what is a group?

There are four properties that define a group:
\begin{enumerate}
    \item \textbf{Closure} under group multiplication. 
        \begin{align}
            g_1\cdot g_2 = g_3 \in G,\; \forall g_1,g_2 \in G
        \end{align}
        So group multiplication is confined to within the bounds of the group.
    \item \textbf{Associativity} of group multiplication. 
        \begin{align}
            g_1\cdot(g_2\cdot g_3) &= (g_1\cdot g_2)\cdot g_3,\; \forall g_1,g_2,g_3 \in G
        \end{align}
    \item \textbf{Identity element}. 
        \begin{align}
            \exists\, e \in G\,:\, e\cdot g = g\cdot e = g,\; \forall g \in G
        \end{align}
    \item \textbf{Inverse element}.
        \begin{align}
             \exists\, g^{-1} \in G\,:\, g^{-1}\cdot g = g\cdot g^{-1} = e,\; \forall g \in G
        \end{align}
\end{enumerate}

Now some notes:
\begin{itemize}
    \item The group is called \textbf{Abelian} iff
        \begin{align}
            g_1\cdot g_2 = g_2\cdot g_1,\; g_1,g_2 \in G
        \end{align}
        This is equivalent to being called commutative, from $[g_1,g_2]=0$.
    \item Then it holds that we have \textbf{non-Abelian} groups, where
        \begin{align}
            g_1\cdot g_2 \neq g_2\cdot g_1
        \end{align}
        This is non-commutative, from $[g_1,g_2]\neq0$.
\end{itemize}
Matrix multiplication (of square matrices) is an example of a non-Abelian group multiplication.

\subsection{Important Examples}
\begin{itemize}
    \item SU(N) is a group of unitary $N\times N$ matrices, with $\text{det}=1$. For SU(N), N is for $N\times N$ matrices, and the U says it is unitary. 
        Unitary is defined by
        \begin{align}
            SU(N) \ni U\,:\, U^\dagger\cdot U = U\cdot U^\dagger = \mathbb{I}_{N\times N},~ U^\dagger = (U^*)^T
        \end{align}
        where $U^\dagger$ is called Hermitian conjugation. 
        The S then defines the group as "Special", which means $\text{det}(U) = 1$.\\
        Let's check these group properties. 
        \begin{enumerate}
            \item Matrix multiplication:
                \begin{align}
                    U_1 &\in SU(N):~ U_1^\dagger U_1 = \mathbb{I} \\
                    U_2 &\in SU(N):~ U_2^\dagger U_2 = \mathbb{I} \\
                    (U_1U_2)^\dagger U_1U_2 &= U_2^\dagger \underbrace{U^\dagger_1 U_1}_{\mathbb{I}}U_2 = U^\dagger_2 U_2 = \mathbb{I} \\
                    \text{det}(U_1U_2) &= \text{det}(U_1)\cdot\text{det}(U_2) = 1
                \end{align}
                So we have \textbf{closure.}
            \item Associativity is satisfied by the definition of matrix multiplication.
            \item Unit matrix:
                \begin{align}
                    e &= \mathbb{I}_{N\times N} 
                \end{align}
            \item The inverse matrix element:
                \begin{align}
                    U^{-1} &= U^\dagger
                \end{align}
        \end{enumerate}
        So we have a group that holds all the properties, a very important group at that. \\
        Consider some general $U \in SU(N)$. 
        How many real independent parameters (real degrees of freedom) does $U$ have?
        An $N\times N$ complex matrix will have $2N^2$ real degrees of freedom. 
        Now if we require unitarity, $U^\dagger = U$, there are $N^2$ constraints on the degrees of freedom, so now we are left with only $N^2$ degrees of freedom by this requirement. 
        Now if we impose that $\text{det}U = 1$, which is a single condition, we are left with $N^2-1$ real degrees of freedom. 
    \item U(1) is a group of unitary $1\times1$ matrices, so $U^\dagger U = 1$. 
        \begin{align}
            \forall U \in U(1)\,:\, U = e^{i\alpha},\; \alpha \in \R
        \end{align}
        Note we cannot require that the $\text{det}U=1$, otherwise we collapse down to a single value of this group, where $\alpha=0$.
        We do not really need to check the group properties of U(1) as they are completely trivial.
    \item SO(N) is a group of $N\times N$ real-valued matrices which are orthogonal:
        \begin{align}
            \forall O \in SO(N)\,:\, O^T\cdot O = O\cdot O^T = \mathbb{I}
        \end{align}
        So N is for $N\times N$, O is for orthogonal, and S again for $\text{det}=1$.\\
        SO(N) matrices are \emph{proper} (we do not do parity transformations of $x\to -x$) rotations in the $\R^N$ (N-dimensional real vector space).
        It is again trivial to find the four group properties fully satisfied for SO(N), so these are groups again.
        \begin{align}
            SO(2) \ni O = \begin{pmatrix} \cos\theta & \sin\theta \\ -\sin\theta & \cos\theta\end{pmatrix}
        \end{align}
        We can see that for SO(2), we have one real parameter in $\theta\in\R,0\leq\theta\leq2\pi$.
\end{itemize}

\chapter{}
Continuing last time, SO(2) is isomorphic to U(1):
\begin{align}
    SO(2):~ O &= \begin{pmatrix}\cos\theta & \sin\theta \\ -\sin\theta & \cos\theta\end{pmatrix} & U(1):~ U &= e^{i\theta} = \cos\theta + i\sin\theta
\end{align}
So both these groups depend only on the value of $\theta$, and knowing any matrix in one of these groups allows us to construct the corresponding matrix in the other.

\section{Group Theory Continued}
\subsection{Direct Products of Groups}
Consider a group $G \ni \{g_1,g_2,\dots\}$, and another group $H \ni \{h_1,h_2,\dots\}$.
We can define a product group from these, $G\times H$, which is also a group. 
The definition of this direct product is by construction.
If we consider a matrix element of matrix $g \in G,\, g_{ij}$, and similarly for in $H$ we consider the matrix element $h_{\alpha\beta}, h \in H$.
Now we construct an object
\begin{align}
    g_{ij}\cdot h_{\alpha\beta} &\equiv (gh)_{i\alpha;j\beta} \\
    G\times H &= \{g_{ij}\cdot h_{\alpha\beta}\}
\end{align}
Let us consider the example of $U(1)\times SU(2)$, where this direct product is equal to $U(2)$, which is a untary group, but the determinant is not $=1$, but $\text{det}=e^{i2\alpha}$, where $\alpha$ was the parameter of $U(1)$.

An important example to keep in mind is the Gauge Field Theory of the Standard Model: $SU(3)\times SU(2)\times U(1)$.

\subsection{Simple vs Non-simple Groups}
\textbf{A simple group} is defined as a group that cannot be written as a direct product of smaller groups, i.e. cannot be decomposed.
A group U(N) is not simple, as
\begin{align}
    U(N) &\approx U(1)\times SU(N),
\end{align}
where SU(N) and U(1) are both trivially simple groups. 

\subsection{Representations of Groups}
We can describe group in two equivalent ways:
\begin{itemize}
    \item A group is some formal mathematical structure - it is some set of elements which satisfies the four definitions of the group and some precise descriptions of what we mean by that.
    \item \textbf{The Fundamental Represenation of the Group} - we can derive a group via an explicit matrix representation, e.g. SU(N) are $N\times N$ complex matrices such that $U^\dagger U=1 = UU^\dagger$ and $\text{det}U=1$.
\end{itemize}
Each group can have many different representations; the fundamental representation is what we used for its definitions.
A group SU(N) in the fundamental representation is given by the $N\times N$ matrices.
These matrices act on some N-dimensional complex vector space described by a N-vector.
\begin{align}
    \begin{pmatrix} \bullet & \bullet & \bullet \\ \bullet & \bullet & \bullet \\ \bullet & \bullet & \bullet \end{pmatrix} \cdot \begin{pmatrix} \bullet \\ \bullet \\ \bullet\end{pmatrix}
\end{align}
An N-vector, $x_i$, is transforming in the fundamental representation of SU(N).

We can also construct a tensor representation,
\begin{align}
    x_i &= \begin{pmatrix} x_1\\ x_2\\x_N\end{pmatrix};y_j = \begin{pmatrix}y_1\\y_2\\y_N\end{pmatrix} \in \C^N:~ x_iy_j = \text{rank-2 tensor, in SU(N)} \\
    \sum_k & \underbrace{U_{ik}}_{\text{SU(N) matrix}}\underbrace{x_k}_{\text{vector}} = \underbrace{x'_i}_{\text{\shortstack{transformed\\ vector}}} \text{ fund. rep.} \\
    \sum_{j'}\sum_{i'} & U_{ii'}U_{jj'}(x_{i'}y_{j'}) = (xy)_{ij}
\end{align}
These rank-two tensor representations can be decomposed into a singlet $\oplus$ traceless symmetric tensor $\oplus$ anti-symmetric tensor representations.
These representations which cannot be reduced any further are called \textbf{irreducible representations.}
So the fundamental representation of any group is irreducible, while rank-two tensor representation is reducible, as said above.

A group element in some general representation can be written as some matrix which can be brought into some block diagonal form, where off-diagonal elements are all zero, where each minimal block along the diagonal is an irreducible representation (irrep) of the group.

\chapter{}
\section{Lie Groups}
For a Lie group, $G$, with an element of this group, $a(\alpha^1,\dots,\alpha^k)$, $a$ depends continuously on parameters $\alpha^1,\dots,\alpha^k$. 
Elements of Lie groups can be represented by
\begin{align}
    a &= e^{-i\sum_{a=1}^k T^a\alpha^a}.
\end{align}
Here, the $\alpha$s are our free parameters, and $T^a$ are the generators of the Lie group, i.e. these are given matrices.
\begin{align}
    \alpha^i=0\; \forall\, i \leq k,\; a = e^0 = \mathbb{I}
\end{align}
We can consider $\alpha^1,\dots,\alpha^k\ll1$ (infinitesimal),
\begin{align}
    a &= e^{-i\sum_{a=1}^k T^a\alpha^a} = \mathbb{I} - i\sum_a T^a\alpha^a + \mathcal{O}(\alpha^2) \\
    T^a &= i\frac{\p a}{\p\alpha^a}\bigg|_{\alpha^a=0}
\end{align}
For example, if $G=SU(2)\ni U_{2\times2}$ (in the fundamental representation):
\begin{align}
    T^b_{2\times2} &= i\frac{\p U_{2\times2}}{\p\alpha^b}
\end{align}
Now back to the general case of a Lie group (\textit{from now on, the sum over repeated indices is assumed}),
\begin{align}
    G \ni a &= \exp\left(-iT^a\alpha_a\right) \\
    T^a &= i\frac{\p a}{\p\alpha^a}\bigg|_{\alpha^a=0}
\end{align}
This will find our generators for the Lie group, but these generators will not commute:
\begin{align}
    [T^a,T^b] &= if^{abc}T^c
\end{align}
This is not an elephant, but another generator with some prefactor, where the $f^{abc}$ is the structure constant of the Lie group. 
Following from the definition of the commutator relation, $f^{abc}$ is completely anti-symmetric around its three indices.

\textbf{Any given Lie group is defined by this relation in Eq (3.8).}
Essentially, the explicit form of the structure constants is what defines any given Lie group as distinct.
From this relation, we can find the set of all generators, $\{T^a\}^k_{a=1}$, which will allow us to write down all elements of our Lie group, $a\in G$ through Eq (3.1).

\subsection{Notations}
Sometimes in the notes, we may use 
\begin{align}
    a &= e^{i\theta^a X_a},~ \theta^a = -\alpha^a,~ X^a = T^a \\
    c^{abc} &=  f^{abc}
\end{align}

\section{Some Simple Lie Groups}
\subsection{U(1) Group}
The simplest example we'll have is U(1) $\ni a$: here, we have $T=1$, and the number of generators is also 1.
        \begin{align}
            a &= e^{-i\alpha\cdot1},~ T=1 \\
            [T,T] &= [1,1] = 0 \implies f^{abc} = 0
        \end{align}
        For all Abelian Lie groups, the commutators are zero (by definition).

\subsection{SU(2) Group}
SU(2) $\ni U$:
\begin{align}
    U &= e^{-i\alpha^a T_a}
\end{align}
So we want to know:
\begin{itemize}
    \item How many $T^a$s are there, i.e. $k$? And what are the generators of SU(2) in the fundamental representation? \\
        We will start by choosing the fundamental representation, where we have $2\times2$ complex matrices with $U^\dagger U = \mathbb{I}$ and $\text{det}U = 1$.
        \begin{align}
            U &= \exp\left(-i\sum_{a=1}^3 \alpha^a\frac{\sigma^a}{2}\right)
        \end{align}
        So we have three real parameters in $\alpha^a$ and the generators, $\frac{\sigma^a}{2}$ are the Pauli matrices (over 2). 
        Is this right?
        Well Pauli matrices are Hermitian, and they provide us with a complete basis of $2\times2$ Hermitian matrices (-1, as it excludes the unit matrix).
        Let's consider the Hermitian conjugate of $U$ to check:
        \begin{align}
            U^\dagger &= \left[\exp\left(-i\alpha^a\frac{\sigma_a}{2}\right)\right]^\dagger = \exp\left(+i\alpha^a\frac{\sigma_a}{2}\right) = U^{-1}
        \end{align}
        So we have unitarity, and we can easily check the other group properties if needed.

        We have learned that the generators of SU(2) (in the fundamental representation) are 
        \begin{align}
            T^a &= \frac{\sigma^a}{2},~ a = \{1,2,3\}
        \end{align}
        This agrees with the argument of free parameters from last lecture where SU(N) has $N^2-1$ free parameters, which for SU(2) requires three free parameters, which we have in our three generators. 
        
        But why is it $\frac{\sigma}{2}$ and not $\sigma$? The $\frac12$ factor is due to normalisation, and depends on how we choose normalisation. 
        In the fundamental representation, we choose
        \begin{align}
            \text{Tr}(T^aT^b) &= \frac12 \delta^{ab},
        \end{align}
        and for $T^a = \frac{\sigma^a}{2}$, we fulfill this requirement.

    \item What is the Lie algebra of SU(2), i.e. $f^{abc}$?
        \begin{align}
            \left[\frac{\sigma^a}{2},\frac{\sigma^b}{2}\right] &= i\e^{abc}\frac{\sigma^c}{2}
        \end{align}
        We can check this directly using the defining properties of Pauli matrices, or just working it out by hand. 
        So our structure constants of SU(2) are $\e^{abc}$, and now we have our full Lie algebra for SU(2):
        \begin{align}
            [T^a,T^b] &= i\e^{abc}T^c
        \end{align}
    \item What about choosing in another representation than the fundamental one?\\
        We can choose any representation of SU(2), and we may get different descriptions of generators, but the Lie group is always defined by Eq (3.8), and for SU(2), $f^{abc} = \e^{abc}$ in any representation, but the simplest one will always be the fundamental one.
\end{itemize}

\chapter{}

\chapter{}
\section{QED}
The Lagrangian of QED:
\begin{align}
    \La = \bar{\psi}(i\gamma^\mu D_\mu - m)\psi - \frac14 F_{\mu\nu}F^{\mu\nu}
\end{align}
QED is a U(1) gauge invariance theory, with Dirac fermion fields $\psi,\bar{\psi}$, and the gauge field $A_\mu$, which is a vector field.
The Dirac fields describe $e^{\pm}$, and the gauge field describes the photons, $\gamma$.

If we consider just the gauge field part of QED:
\begin{align}
    \La[A_\mu] &= -\frac14 F^{\mu\nu}F_{\mu\nu}
\end{align} 
We can then begin to write down equations of motion from this, first considering the action.
\begin{align}
    \mathcal{S} &= \int \La[A_\mu]\,d^4x
\end{align}
We take the extremum of the action, where $\frac{\delta\mathcal{S}[A]}{\delta A_\nu}=0$, to find the Euler-Lagrange equations. 
\begin{align}
    \p_\mu \frac{\p\La[A]}{\p(\p_\mu A_\nu)} &= \frac{\p\La[A]}{\p A_\nu} = 0 \\
    \p_\mu F^{\mu\nu} &= 0
\end{align}
However, if we look at the full Lagrangian again and fully express $D_\mu = \p_\mu + ieA_\mu$, then Eq (5.5) is no longer equal to zero when fermions are present.
\begin{align}
    \p_\mu F^{\mu\nu} &= e\bar{\psi}\gamma^\nu\psi \equiv j^\nu
\end{align}
These are two of the Maxwell equations (out of four) for QED.
The other two Maxwell equations are trivial in QED, following from the Bianchi identity, 
\begin{align}
    \p_\lambda F_{\mu\nu} + \p_\mu F_{\nu\lambda} + \p_\nu F_{\lambda\mu} = 0.
\end{align}
This identity is automatically satisfied for $F_{\mu\nu} = \p_\mu A_\nu - \p_\nu A_\mu$.
So all four classical Maxwell equations can be written in QED as 
\begin{align}
    \frac{\p S}{\p A_\nu} = 0 &\implies \p_\mu F^{\mu\nu} = j^\nu \\
    \frac{\p S}{\p \bar{\psi}} = 0 &\implies (i\gamma^\mu D_\mu - m)\psi = 0  
\end{align}
Eq (5.9) is the Dirac equation with the field $A_\mu$ in the $D_\mu$ term. 

\subsection{How many degrees of freedom does the photon field have?}
Naively, it looks like it has four degrees of freedom, as we have $A_\nu, \nu = 0,1,2,3$. 
In reality, there are only two physical degrees of freedom of the photon.
Why is that?
$A_0$ field decouples $\to$ it is not a dynamical field as it does not have a kinetic term, i.e. $\frac12(\p_t A_0(x))^2$ is absent in the Lagrangian, but this term is required for any fieldto be kinetic as it would describe velocity.\\
We can always fix the gauge freedom by setting $A_0\equiv0$.
We can further set $\p_iA_i=0$ - this is the Coulomb gauge.
So we have two constraints from fixing the gauge, so two \emph{unphysical} degrees of freedom are removed, leaving us with $4-2=2$ physical degrees of freedom (assuming unbroken gauge invariance).
These 2 degrees of freedom of the photon are its 2 transverse polarisations. 

$A_\mu$ describes spin-1 fields (or particles), and we have the two gauges of $A_0 = 0$ and $\p_i A_i = 0 \implies p_i A_i = 0$, where $p_i$ is the three-momenta.
If we then choose momentum to be wholly along the z-coordinate, so $\vec{p}=(0,0,p)$, then $\vec{A} = (A_1,A_2,0)$, and then we see that we have two transverse polarisations along x and y, while the momentum transfer is all in z - this is the simplest choice for example but any physical orientation of $\vec{p}$ and $\vec{A}$ will lead to the same two transverse polarisations.

\subsection{What is allowed in QED?}
Again, we write down the QED Lagrangian:
\begin{align}
    \La_{QED} = &-\frac14 F^{\mu\nu}F_{\mu\nu} \text{ - propagator of the photon field $A_\mu$, quadratic in it} \nonumber \\
                &+ \bar{\psi}(i\gamma^\mu\p_\mu -m)\psi \text{ - free propagation of fermions} \nonumber \\
                &- e\bar{\psi}\gamma^\mu A_\mu\psi \text{ - 3 point vertex describing interactions}
\end{align}
$\La_{QED}$ is uniquely constructed from the requirement of gauge invariance. 
Can we add other gauge-invariant interactions?
Consider
\begin{align}
    \La &= F_{\mu\nu}F^{\nu\alpha}F_\alpha^\mu
\end{align}
\textit{fix}
For mass dimension, $[]$: we have
\begin{align}
    [A_\mu] &= 1 & [\phi] & = 1 & [\La] &= 4 \\
    [\psi] &= \frac32 & [\bar{\psi}] &= \frac32 & [\mathcal{S}] &= 0 
\end{align}
So we can see the mass dimension of the above $\La$ will be 6. 
We could add it to the QED Lagrangian with some pre-factor to get it to work?
\begin{align}
    \La &= \La_{QED} + \frac{1}{M^2}F_{\mu\nu}F^{\nu\alpha}F_\alpha^\mu
\end{align}
Any terms of $\La$ that have a coefficient of negative mass dimension in front are not UV-renormalisable, and any operators is $\La$ of mass dimension greater than 4 are not renormalisable. 

\subsection{What is UV Renormalisation?}
Any Quantum Field Theory which constains quantum corrections such as
\textit{feynman diagrams - tree level, then one loop correction and up to higher loops}
All loop-level corrections contain $\infty$ in the UV, so we need to have a prescription to remove this divergence.

\textbf{UV Renormalisation} is the prescription to remove UV divergences.

So now $\La_{QED}$ is completed and is shown that we cannot remove or add anything from/to it. 

\end{document}



















