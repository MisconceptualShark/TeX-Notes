\documentclass[a4paper, 11pt, normalem]{report}

\usepackage{../../../LaTeX-Templates/Notes}
\usepackage{subfiles}

\title{Foundations of Physics 3A \vspace{-20pt}}
\author{Dr Nikitas Gidopoulos}
\date{\vspace{-15pt}Michaelmas Term 2018}
\rhead{\hyperlink{page.1}{Contents}}

\begin{document}

\maketitle
\tableofcontents

\part{Quantum Mechanics}
\chapter{}

1. Hilbert spaces
2. von Neumanns axioms
3. Hermitian ops
4. T.D. Schrodinger Eq -> T.I. Schrodinger Eq
5. 1 particle in 3D
6. Ang momentum ops, commutators, spin angular momentumo

\section{von Neumanns axioms}

1. In QM, every observable is represented by a Hermitian operator.
2. The state of the system is represented by a wavefunction.
    - Scalar multiple of the wavefunction still represents system, convenient to choose to be normalised though.
3. Usual bra-ket stuff
\begin{align}
    \langle\psi|Q\psi\rangle &= \int dx \psi^{*}(x)\left(Q\psi(x)\right) \\
    \langle\psi|Q\psi\rangle^{*} = \langle Q\psi|\psi\rangle &= \int dx \left(Q\psi(x)\right)^{*}\psi(x) \\
    \langle Q\rangle &= \langle\psi|Q\psi\rangle = \langle Q\psi|\psi\rangle = \langle\psi|Q|\psi\rangle
\end{align}

4. Usual orthonormal eigen stuff

\subsection{Example}

See Pauli spin operators and stuff
Find result of $\omega = \gamma B$, the angular momentum around a magnetic field

\chapter{}

nothing

\chapter{}

1. Study motion of charged particle in magnetic fields
2. Study means to solve S.E. - we need H for B, E, $H = \frac{1}{2m}\left(-i\hbar\nabla - qA\right)^2 + q\phi$
3. It turns out H contains $\phi, A$, not E, B, $E = -\nabla\phi$, $B = \nabla\times A$
4. But different As may gave same B
5. Solve examples
    1. 2D motion of q in uniform mag field (cyclotron freq)
    2. Explore A vs B - 1D example where $B = 0, A \neq 0$

\section{Guage freedom}
\begin{align}
    \vec{A} &= \frac{B}{2}(-y\hat{x} + x\hat{y} \implies \vec{\nabla} \times\vec{A} = B\hat{z} \\
    \vec{A}' &= -By\hat{x} \implies \vec{\nabla}\times\vec{A}' = B\hat{z} \\
    \vec{A} - \vec{A}' &= \frac{By}{2}\hat{x} + \frac{Bx}{2}\hat{y} = \vec{\nabla}\left(-\frac{Bxy}{2}\right) \implies \vec{A} = \vec{A}' + \vec{\nabla}\times(x,y,z) \\
    \vec{\nabla} \times \vec{A} &= \vec{\nabla} \times \vec{A} + \cancel{\vec{\nabla}} \times \vec{\nabla}\times(\vec{r})
\end{align}

\subsection{Time deriv of fn}
\begin{align}
    \phi' &= \phi + \frac{\partial}{\partial t}\times(r,t) \\
    \vec{A}' &= \vec{A} + \vec{\nabla}\times(r,t) \\
    (\phi',\vec{A}'),& (\phi,\vec{A}) \to \vec{E},\vec{B}
\end{align}
Convenient to choose $\vec{A}$ s.t. $\vec{\nabla}\cdot\vec{A} = 0$.

\section{2D motion (x,y) of q in mag field, B along z}

\begin{align}
    H &= \frac{1}{2m}\left(-i\hbar\vec{\nabla} - q\vec{A}\right)^2 \\
    &= \frac{1}{2m}\left(-i\hbar\frac{\partial}{\partial x}\hat{x} - i\hbar\frac{\partial}{\partial y}\hat{y} + qBy\hat{x}\right)^2 \\
    &= \frac{1}{2m}\left(-i\hbar\frac{\partial}{\partial y}\hat{y} + \left(-i\hbar\frac{\partial}{\partial x} + qBy\right)\hat{x}\right)^2 \\
    E\psi(x,y) &= \frac{1}{2m}\left(-\hbar^2\frac{\partial}{\partial x^2} - 2i\hbar qBy\frac{\partial}{\partial x} + q^2B^2y^2 - \hbar^2\frac{\partial^2}{\partial y^2}\right)\psi(x,y) \\
    \psi(x,y) &= e^{-ikx}\phi(y) \\
    \implies &\frac{1}{2m}\left(\hbar^2k^2 - 2\hbar kqBy + q^2B^2y^2 - \hbar^2\frac{\partial^2}{\partial y^2}\right)\cancel{e^{-ikx}}\phi(y) = E\cancel{e^{-ikx}}\phi(y)  \\
    \implies &\frac{1}{2m}(qBy - \hbar k)^2\phi(y) = E\phi(y), y_0 = \frac{\hbar k}{qB} \\
    \implies &\left[-\frac{\hbar^2}{2m}\frac{\partial^2}{\partial y^2} + \frac{m}{2}\frac{q^2B^2}{m^2}(y - y_0)^2\right]\psi(y) = E\psi(y)
\end{align}
This is Simple harmonic motion with $\frac{qB}{m} = \omega$, and energy $E = \left(n + \frac{1}{2}\right)\hbar\omega$.
These are the Landau levels.

\subsection{Periodic boundary conditions}

\begin{align}
    e^{-ik(x + L)} = e^{-ikx} \\
    e^{-ikL} = 1 \\
    kL = 2\pi j \\
    k_j = \frac{2\pi}{L}j, j \in Z \\
    0 < y_0 < L \\
    0 < \frac{\hbar k_j}{qB} < L \\
    0 < \frac{\hbar 2\pi j}{qBL} < L \\
    0 < j < \frac{qBL^2}{h} \\
    0 < j < \frac{\Phi}{h/q} \\
    j_{max} = \frac{\Phi}{h/q}
\end{align}

\section{A vs B, Aharanov-Bohm}

\begin{align}
    H = \frac{L_z^2}{2I} &= -\frac{\hbar^2}{2mb^2}\frac{\partial^2}{\partial \phi^2} \\
    -frac{\hbar^2}{2mb^2}\frac{\partial^2}{\partial \phi^2}\psi(\phi) &= E\psi(\phi) \\
    E &= \frac{\hbar^2 k^2}{2mb^2}
\end{align}

Single valued: $e^{ik\phi}$ soln, $e^{ik(\phi + 2\pi)} = e^{ik\phi}$. k is integer.

\chapter{}
\section{Particle moving on ring, radius b in xy plane}

Without magnetic field:
\begin{align}
    H &= -\frac{\hbar^2}{2mb^2}\frac{\partial^2}{\partial\phi^2} \\
    \implies E\Psi(\phi) &= -\frac{\hbar^2}{2mb^2}\frac{\partial^2}{\partial\phi^2}\Psi(\phi) \\
    \frac{L_z^2}{2I}, \Psi(\phi) &= \frac{e^{2n\phi}}{\sqrt{2\pi}}, 0 \leq \phi \leq 2\pi \\
    e^{in(\phi+2\pi)} &= e^{in\phi} \\
    \implies e^{in2\pi} &= 1 \implies n \in \mathbb{Z} \\
    \Psi_n(\phi) &= \frac{e^{in\phi}}{\sqrt{2\pi}} \\
    E_n &= \frac{\hbar^2n^2}{2mb^2}
\end{align}

With magnetic field:
\begin{align}
    \vec{A} &= \begin{cases} \frac{\Phi\dot\rho}{2\pi a^2}\hat{\phi} & \rho < a \\ \frac{\Phi}{2\pi a}\hat{\phi} & \rho > a \end{cases} \\
    \rho < a, \vec{\nabla}\times\vec{A} &= \frac{\hat{z}}{\rho} \frac{\partial}{\partial \rho}(\rho A_\phi) \\
    &= \hat{z}\frac{\Phi}{\pi a^2} = \hat{z}B
    \rho > a, \vec{\nabla}\times\vec{A} &= 0 \implies B = 0 \\
    H &= \frac{1}{2m}\left[\left(-i\hbar\frac{1}{\rho}\frac{\partial}{\partial\phi} - q\frac{\Phi}{2\pi a^2}\right)\hat{\phi}\right]^2 \\
    &= \frac{\hbar^2}{2mb^2}\left[-\frac{\partial^2}{\partial\phi^2} + 2i\frac{q\Phi}{h}\frac{\partial}{\partial\phi} + \left(\frac{q\Phi}{h}\right)^2\right] \\
    H\Psi(\phi) &= E\Psi(\phi) \\
    E_n &= \frac{\hbar^2}{2mb^2}\left[n^2 - 2\frac{q\Phi n}{h} + \left(\frac{q\Phi}{h}\right)^2\right] \\
    &= \frac{\hbar^2}{2mb^2}\left(n - \frac{q\Phi}{h}\right)^2
\end{align}
Energy will be parabolas with minimum of zero on a plot of flux and energy.
Increasing n will shift the centre of parabola along axis.

\section{Guage Invariance}

1. The same $\vec{E},\vec{B}$ can be given from different $\phi,\vec{A}$

\begin{align}
    \vec{E} &= -\vec{\nabla} - \frac{\partial}{\partial t}\vec{A} \\
    \vec{B} &= \vec{\nabla}\times\vec{A} \\
    \phi' &= \phi - \frac{\partial A}{\partial t} \\
    \vec{A}' &= \vec{A} + \vec{\nabla}A
\end{align}

2. Same $\vec{E},\vec{B}$ have different H, so different wavefns

3. Do measurable quantities depend on choice of guage?

\begin{align}
    \rho(r,t) &= |\Psi(r,t)|^2 \\
    j(r,t) &= \text{ probability current density} \\
    \frac{\partial\rho(r,t)}{\partial t} + \vec{\nabla}\dot\vec{j}(r,t) = 0 \\
    \int dx \left[\frac{\partial\rho}{\partial t} + \frac{d}{dx}j(x)\right] &= \frac{\partial}{\partial t} \int_A^B \rho(x)\,dx + \int_A^B \frac{d}{dx}j\,dx = 0 \\
    \frac{\partial}{\partial t}N_{AB} = j(A) - j(B)
\end{align}

\chapter{}
\section{Equation for conservation of probability}
\begin{align}
    \Psi^*(\vr)i\hbar\frac{\p\Psi(\vr,t)}{\p t} &= \Psi^*(\vr)\left(\frac{-\hbar^2}{2m}\right)\left(\del^2\Psi(\vr,t)\right) + \Psi^*(\vr)V(r)\Psi(\vr,t) \\
    i\hbar\left[\Psi^*\frac{\p\Psi}{\p t} + \Psi\frac{\p\Psi^*}{\p t}\right] &= -\frac{\hbar^2}{2m}\left[\Psi^*\left(\del^2\Psi\right) - \Psi\left(\del^2\Psi^*\right)\right] \\
                                                                             &= \frac{i\hbar i\hbar}{2m}\del\cdot\left[\Psi^*\left(\del\Psi\right) - \Psi\left(\del\Psi^*\right)\right] \\
    \frac{\p}{\p t}\rho(\vr,t) &+ \del\left(\frac{-i\hbar}{2m}\right)\left[\Psi\left(\del\Psi\right) - \Psi\left(\del\Psi^*\right)\right] = 0
\end{align}

\section{Rayleigh-Ritz variational principle}
\begin{enumerate}
    \item Approximation method (useful and powerful). Find approx. for solution of S.E.
    \item Powerful theoretical tool/concept. Equivalent to S.E.
\end{enumerate}

For one particle,
\begin{align}
    H &= -\frac{\hbar^2}{2m}\del^2 + V(r) \\
    H&\Psi_n(\vr) = E_n\Psi_n(\vr) \\
    \Psi(\vr) &= \sum_n c_n\Psi_n(\vr) \\
    \Psi^*(\vr) &= \sum_m c^*_m\Psi^*_m(\vr) \\
    \langle\Psi|H|\Psi\rangle &\geq E_0 \\
                              &= \int d^3r \Psi^*(r) H \Psi(r) = \int d^3r \sum_m c^*_m\Psi^*(r) H \sum_nc_n\Psi_n(r) \\
                              &= \sum_{m,n} c_m^*c_n \int d^3r \Psi^*_m(r) \left(H\Psi_n(r)\right) = \sum_{m,n} c_m^*c_nE_n \int d^3r \Psi^*_m(r)\Psi_n(r) \\
    \langle\Psi|H|\Psi\rangle &= \sum_n |c_n|^2E_n \delta_{nm} = |c_0|^2E_0 + |c_1|^2E_1 + \dots \geq E_0\sum_n |c_n|^2 \\
    \implies \langle\Psi|H|\Psi\rangle &\geq E_0\sum_n |c_n|^2 \\
                                       &\geq E_0 ~\forall ~ \Psi
\end{align}
For Lagrange multipliers, see notes on DUO.

\chapter{}
Find the minimum of
\begin{align}
    G[\phi] &= \left[\langle\phi|H|\phi\rangle - \lambda\langle\phi|\phi\rangle\right] \\
    \lim_{\e\to 0}& \frac{G[\phi + \e u] - G[\phi]}{\e} = 0, ~\forall\; u(\vr) \\
    G[\phi + \e u] &= \langle\phi+\e u|H|\phi+\e u\rangle - \lambda\langle\phi+\e u|\phi+\e u\rangle \\
                   &= \int dr (\phi^*(\vr) + \e u^*(\vr))H(\phi(\vr)+\e u(\vr)) - \lambda\int dr |\phi(\vr) + \e u(\vr)|^2 \\
    \begin{split}
                   &= \left[\int d^3r \phi^*(\vr)H\phi(\vr) - \lambda\int dr |\phi(r)|^2\right] + \e\left[\int dr \;u^*(\vr)H\phi(\vr) - \lambda\int dr\; u^*(\vr)\phi(\vr)\right] \\
                   &+ \e\left[\int dr \phi^*(\vr)Hu(\vr) - \lambda\int dr\;\phi^*(\vr)u(\vr)\right]
    \end{split} \\
                   &= G[\phi] + \e(\langle u|H\phi\rangle -\lambda\langle u|\phi\rangle) + \e(\langle\phi|H|u\rangle -\lambda\langle\phi|u\rangle) + O(\e^2) \\
    \frac{G[\phi+\e u] - G[\phi]}{\e} &= \lim_{\e\to 0} \frac{\e(\langle u|H|\phi\rangle - \lambda\langle u|\phi\rangle) + \e(\langle\phi|H|u\rangle - \lambda\langle\phi|u\rangle)}{\e} \\
    \implies& \int dr\;u^*(\vr)[H\phi(\vr) - \lambda\phi(\vr)] = 0 \\
    \implies& H\phi(\vr) - \lambda\phi(\vr) = 0
\end{align}

\chapter{}
\textit{there may be one or two missing lectures here}

\begin{enumerate}
    \item Example - 2 electrons without repulsion, ground and first excited states
    \item Switch on electron interaction
    \item N electrons, no interaction - Slater determinants
    \item Introduce interaction, electron strucutre theory, many body
    \item mean field, Hartree-Fock
\end{enumerate}


\end{document}
