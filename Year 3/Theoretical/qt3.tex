\documentclass[a4paper, 11pt, normalem]{report}

\usepackage{../../../LaTeX-Templates/Notes}
\usepackage{subfiles}

\title{Quantum Theory 3 \vspace{-20pt}}
\author{Prof Khoze}
\date{\vspace{-15pt}Epiphany Term 2019}
\rhead{\hyperlink{page.1}{Contents}}

\newcommand{\dens}{\hat{\rho}}
\newcommand{\fp}{\cancel{\p}}

\begin{document}

\maketitle
\tableofcontents

\part{Scattering Theory}
\chapter{Introduction to Scattering}
\section{Two types of scattering}
\begin{itemize}
    \item elastic - initial particles remain and no new particles emerge in the collision
    \item inelastic - in the final state, there is more than just the initial particles
\end{itemize}
Will be using \unl{non-relativistic} Quantum Mechanics for this part of the course, therefore will only be studying elastic non-relativistic scattering, e.g. Rutherford experiment, $\alpha + \text{Au} \to \alpha + \text{Au}$.
\begin{itemize}
    \item Consider elastic $e^-e^+$ scattering
        \begin{equation}
            e^+ + e^- \to e^+ + e^-
        \end{equation}
        \emph{Feynman diagrams of collision - both s-channel (particles meet) and t-channel (particles interact through virtual photon).}
    \item Consider inelastic scattering
        \begin{equation}
            e^+ + e^- \to \mu^+ + \mu^-
        \end{equation}
        \emph{Feynman diagram of collision and decay into muon and anti-muon - electron and positron collide and annihilate, their energy then carried by photon which decays into muon and anti-muon.}
\end{itemize}

\section{Scattering cross-sections}
The scattering cross-section, $\sigma$, is a probabilistic quantity that characterises the 'strength' of the scattering (interaction between the particles).
$\sigma$ has dimension of area ($m^2$).
\begin{align}
    \frac{d\sigma}{d\Omega} &= \frac{1}{F} \frac{dR}{d\Omega},
\end{align}
where:
\begin{itemize}
    \item $F$ is the flux - number of incident particles per unit area per unit time ($s^{-1}m^{-2}$)
    \item $dR$ is the rate - number of scattered particles ($N$) into $d\Omega$ per unit time ($s^{-1}$)
    \item $d\Omega$ is the solid angle
    \item $\frac{d\sigma}{d\Omega}$ is the differential cross-section into the solid angle ($m^2$)
\end{itemize}
\begin{align}
    \frac{d\sigma}{d\Omega}(\theta,\phi) &= \frac{1}{F}\frac{dR}{d\Omega} \\
    \sigma_{tot} &= \int_0^{2\pi} d\phi \int_0^\pi d\theta \sin\theta \frac{d\sigma}{d\Omega}(\theta,\phi) \\
    N &= \sigma_{tot} \cdot \int F\,dt
\end{align}
\begin{itemize}
    \item $F$ is known for each experiment, part of the design
    \item $\sigma_{tot}$ is measured in experiment
        \begin{itemize}
            \item Measured in barns (b) - $1 \text{ barn} = 10^{-24} cm^{-2}$
            \item $\sigma_{\text{Thompson}} = 0.665\,b$
        \end{itemize}
    \item LHC gluon fusion into Higgs:
        \begin{itemize}
            \item $g+g \to H$
            \item \emph{Feynman diagram of gluon collision into Higgs boson}
            \item $\sigma_{\text{Higgs}} \approx 10 pb$
        \end{itemize}
\end{itemize}

\chapter{General Features of Potential Scattering in QM}
\section{The Schrodinger Equation}
Time-dependent Schrodinger equation, and reduced mass:
\begin{align}
    i\hbar\frac{\p\psi(r,t)}{\p t} &= \left(-\frac{\hbar^2}{2m}\del^2 + V(r)\right)\psi(r,t) \\
    m &= \frac{m_Am_B}{m_A+m_B}
\end{align}
$E=$ fixed and finite
\begin{align}
    E&=\frac{p^2}{2m} \\
    \psi(r,t) &= e^{-iEt/\hbar}\psi(r)
\end{align}
Leads to time-independent Schrodinger equation
\begin{align}
    E\psi(r) &= \left(-\frac{\hbar^2}{2m}\del^2 + V(r)\right)\psi(r) 
\end{align}
$\frac{p}{\hbar} = k$, $U(r) = \frac{2m}{\hbar^2}V(r)\to$ Scattering equation:
\begin{equation}
    \left(\del^2 + k^2 - U(r)\right)\psi(r) = 0
\end{equation}
For scattering:
\begin{itemize}
    \item Looking for $\psi(r)$ s.t. as $r \to \pm \infty$, 
        \begin{align}
            \psi_{inc}(r) + \psi_{scat}(r) &\equiv e^{ik\cdot r} + \frac{e^{i|k|\cdot|r|}}{r}\cdot f(k,\theta,\phi) \\
                                           &= e^{ikr\cos\theta} + \frac{e^{ikr}}{r}\cdot f(k,\theta,\phi)
        \end{align}
    \item When scattering occurs, the incoming plane waves turn into spherical waves with $\frac{1}{r}$ amplitude from point of scattering
    \item $f(k,\theta,\phi)$ is the scattering amplitude - need to determine this in order to compute $\sigma$
\end{itemize}
\begin{equation}
    \psi(r) \approx_{r \to \infty} 1\cdot e^{ik\cdot r} + f(k,\theta,\phi)\frac{e^{ik\cdot r}}{r}
\end{equation}
Now consider the probability density for normalisation,
\begin{align}
    \rho_{inc}(r) &\equiv |\psi_{inc}|^2 = 1
\end{align}
What about flux?
\begin{align}
    F &= \frac{\text{\# of incoming particles}}{\text{Area}\cdot\text{time}} \\ 
      &= v\cdot\rho = \frac{p}{m}\cdot\rho,~ \rho = 1 \\
      &= \frac{p}{m}
\end{align}
Recall $\frac{d\sigma}{d\Omega} = \frac{1}{F}\frac{dR}{d\Omega}$. So what is $dR$?
\begin{align}
    dR &= j_r r^2 d\Omega
\end{align}
So $j_r$ is the probability current density - the number of scattered particles crossing the unit area per unit time.
\begin{align}
    j_r &\equiv \frac{\hbar}{2mi}\left(\psi^*_{scat}(r)\del\psi_{scat}(r) - \left(\del\psi_{scat}(r)\right)^*\psi(r)\right) \\
        &= \frac{\hbar}{m}\text{Im}\left(\psi^*_{scat}\del\psi_{scat}\right) = \frac{\hbar k}{m}\frac{|f|^2}{r^2} \\
        &= \frac{p}{m} |f|^2\frac{1}{r^2} = F\frac{|f(k,\theta,\phi)|^2}{r^2} \\
    \implies \frac{d\sigma}{d\Omega} &= \frac{1}{F}\frac{dR}{d\Omega} = |f(k,\theta,\phi)|^2
\end{align}


\chapter{Special Functions for central potentials}
Scattering equation:
\begin{equation}
    \left(\del^2 + k^2 + U(\vr)\right)\psi_k(\vr) = 0
\end{equation}
This is the Schrodinger equation in terms of the rescaled variables - $k = \frac{p}{\hbar}$, $U(\vr) = \frac{2m}{\hbar^2}V(\vr)$.
Assume that the potential is central, $U(\vr) = U(r)$.
So we will use spherical polar coordinates. 
\begin{align}
    \bigtriangleup \equiv \vec{\del}^2 = \frac{1}{r^2}\frac{\p}{\p r}\left(r^2\frac{\p}{\p r}\right) &+ \underbrace{\frac{1}{r^2\sin\theta}\frac{\p}{\p\theta}\left(\sin\theta\frac{\p}{\p\theta}\right) + \frac{1}{r^2\sin^2\theta}\frac{\p^2}{\p\phi^2}}_{} \\
    &\implies -\frac{L^2(\theta,\phi)}{\hbar^2r^2} 
\end{align}
$L^2$ is the operator of the angular momentum squared. 
Use the separation of variables method to split wavefunction into radial and angular components.
\begin{align}
    \psi_k(\vr) &= \sum_{l,m} c_{klm} R_{klm}(r)Y_{lm}(\theta,\phi)
\end{align}
What is $R$ and what is $Y$?
\begin{align}
    L^2Y_{lm}(\theta,\phi) &= l(l+1)\hbar^2 Y_{lm}(\theta,\phi)
\end{align}
So $Y_{lm}$ is an eigenfunction of $L^2$ with eigenvalue of $l(l+1)\hbar^2$.
$Y_{lm}$ are known as the spherical harmonics.
They are also eigenfunctions of the operator, $L_z$.
\begin{align}
    L_zY_{lm} &= m\hbar Y_{lm} \\
    l\in\Z &, -l \leq m \leq l \\
    Y_{lm}(\theta,\phi) &= C_{lm}e^{im\phi}P_l^m(\cos\theta) \\
    P_l^m(x) &= (1-x^2)^{|m|/2}\left(\frac{d}{dx}\right)^|m| P_l(x) \\
    P_l(x) &= \frac{1}{2^ll!}\left(\frac{d}{dx}\right)^l (x^2-1)^2
\end{align}
$P_l^m(x)$ is the associated Legendre polynomial, and $P_l(x)$ is the regular Legendre polynomial.
Spherical harmonics give a complete orthonormal set of functions of $\theta,\phi$.
\begin{align}
    \int Y^m_lY_{l'}^{m'*} d\Omega &= \delta_{ll'}\delta_{mm'} \\
    \lim_{m\to0} Y_{lm} &= \alpha P_l(\cos\theta)
\end{align}
(3.12) is known as azimuthal symmetry.
\begin{align}
    \frac{L^2}{-\hbar^2r^2}Y^m_l(\theta,\phi) = -\frac{1}{\hbar^2r^2}\hbar^2 l(l+1)Y^m_l(\theta,\phi) \\
    \psi_k(\vr) = \sum_{l=0}^\infty \sum_{m=-l}^{+l} c_{klm} R_{klm}(r)Y_{lm}(\theta,\phi) \\
    \left(\frac{\p^2}{\p r^2} + \frac{2}{r}\frac{\p}{\p r} - \frac{l(l+1)}{r^2} - U(r) + k^2\right)R_{klm}(r) = 0
\end{align}
(3.15) is the equation for the radial component, $R_{klm}(r)$, solve to determine.
Assune the potential, $U(r)$, has a finite range, $\lim_{r\to\infty}U(r) = 0$.
\begin{align}
    \lim_{r\to\infty} \psi_k(\vr) = e^{ik\cdot \vr} + f(k,\theta,\phi)\frac{e^{ikr}}{r}
\end{align}
This is our boundary (asymptotic) condition.

To start simply,
\begin{enumerate}
    \item $r\to\infty$, $U(r) \to 0$
    \item $l=0$
\end{enumerate}
For this case, we are solving
\begin{align}
    \left(\frac{\p^2}{\p r^2} + \frac{2}{r}\frac{\p}{\p r} - \cancel{\frac{l(l+1)}{r^2}} + \cancel{U(r)} + k^2\right)R_{k00} = 0
\end{align}
This has a general solution:
\begin{align}
    R_{k00} &= A\cdot\frac{\sin(kr)}{r} + B\cdot\frac{\cos(kr)}{r} \\
    R_{k00} &= \frac{1}{r}U(r) 
\end{align}
A, B are arbitrary constants.
Substitute (3.19) into (3.17) will give (3.18), the simplest of cases where the full function will follow using spherical harmonics.

\chapter{}

\chapter{The Method of Partial Waves}
\begin{align}
    \psi_k(r,\theta,\phi) &\to_{r\to\infty} e^{ik\cdot r} + f(k,\theta, \phi)\frac{e^{ikr}}{r}
\end{align}
$f(k,\theta,\phi)$ is the scattering amplitude. 
Usin $V(r)$ and assume azimuthal symmetry - no dependence on $\phi$.
\begin{align}
    Y_l^m(\theta,\phi=0) &\implies y_l^0(\theta,\cancel{\phi}) = P_l(\cos\theta) \\
    \psi_k(r,\theta) &\to_{r\to\infty} \sum_{l=0}^\infty R_{kl}(r)P_l(\cos\theta) \\
    f(k,\theta) &\to_{r\to\infty} \sum_{l=0}^\infty f_l(k)P_l(\cos\theta)
\end{align}
Each term in the sum of (5.3) is known as a partial wave, so it is a sum over partial waves.
So we know the Legendre polynomials, and must determine the amplitude of the partial waves.
\begin{align}
    \frac{d\sigma}{d\Omega} &= |f(l,\theta,\phi)|^2 \\
                            &= \sum_{l'=0}^\infty\sum_{l=0}^\infty f_{l'}^*(k)f_l(k) P_{l'}(\cos\theta)P_l(\cos\theta) \\
    \sigma &= \int \frac{d\sigma}{d\Omega}d\Omega = 2\pi \int \frac{d\sigma}{d\Omega}\,d(\cos\theta)
\end{align}
Note that
\begin{align}
    \int_{-1}^{+1} P_{l'}(x)P_l(x) \,dx &= \frac{2}{2l+1}\delta_{l'l} \\
    \implies \sigma_{tot} &= 4\pi \sum_{l=0}^\infty \frac{1}{2l+1}|f_l(k)|^2 \\
                          &= \sum_{l=0}^\infty \sigma_l,~~ \sigma_l = \frac{4\pi}{2l+1}|f_l|^2
\end{align}
This is the partial wave decomposition of the total cross-section. 
$\sigma_l$ is the cross-section for the $l$-th partial wave.
$f_l$ is still our unknown to be determined.
\begin{align}
    R_{kl}(r) &\to_{r\to\infty} (2l+1) i^l j_l(kr) + f_l(kr)\frac{e^{ikr}}{r}
\end{align}
The first term comes from the Rayleigh formula; the second term just comes from above.
At the same time, we know that we can express the radial wavefunction using the spherical Bessel functions:
\begin{align}
    R_{kl}(r) &= B_l(k)j_l(kr) + C_l(k)n_l(kr)
\end{align}
$B_l$ and $C_l$ are r-independent constants, $j_l$ and $n_l$ are spherical Bessel functions, first and second kind respectively.
\begin{align}
    R_{kl}(r) &\to_{r\to\infty} \frac{e^{i\alpha}}{kr}\left[b_l(k)\sin\left(kr - \frac{\pi l}{2}\right) - c_l(k)\cos\left(kr-\frac{\pi l}{2}\right)\right] \\
    \delta_l(k) &= -\arctan\left(\frac{c_l(k)}{b_l(k)}\right) \\
    A_l(k) &= \sqrt{b_l^2(k) + c_l^2(k)}
\end{align}
$\delta_l(k)$ is the phase shift - all the non-trivial information about scattering is contained in $\delta_l(k)$.
\begin{align}  
    R_{kl}(r) &= e^{i\alpha}A_l(k)\frac{1}{kr}\sin\left(kr - \frac{\pi l}{2} + \delta_l(k)\right)
\end{align}
Now comparing (5.11) with (5.16):
\begin{align}
    A_l(kr)e^{i\alpha} &= (2l+1)i^le^{\delta_l(k)} \\
    f_l(k) &= \frac{2l+1}{k}e^{i\delta_l(k)}\sin(\delta_l(k))
\end{align}
The partial l-wave scattering amplitude is now known in terms of $\delta_l(k)$, the phase shifts.
This allows us to express the total cross-section in terms of the phase shifts. 
\begin{align}
    |f_l(k)|^2 &= \frac{(2l+1)^2}{k^2}\sin^2(\delta_l(k)) \\
    \sigma_l &= \frac{4\pi(2l+1)}{k^2}\sin^2(\delta_l(k)) \\
    \sigma &= \sum_{l=0}^\infty \frac{4\pi}{k^2}(2l+1)\sin^2(\delta_l(k))
\end{align}
Consider the trivial example of zero potential $\implies$ no scattering.
\begin{equation}
    \delta_l(k) = 0
\end{equation}
This comes from there being no second kind Bessel function as it is singular at the origin.
The cross-section follows from this as equal to 0 from (5.21).

\chapter{}
\begin{equation}
    \lim_{r\to\infty} R_{kl}(r) = \sum_{l=0}^\infty \left(B_l(k)j_l(kr) + C_l(k)n_l(kr)\right)
\end{equation}
$B_l(k)$ and $C_l(k)$ are constants, no r-dependence.
To conserve probability at $r\to\infty$, they must not have a relative phase, i.e.
\begin{equation}
    \frac{B_l(k)}{C_l(k)} = \text{real.}
\end{equation}

\begin{align}
    \lim_{r\to\infty} R_{kl}(r) &= (2l+1)i^l\sin\left(kr - \frac{\pi l}{2}\right) + f_l(k)\frac{e^{ikr}}{r} \\
    \lim_{r\to\infty} \psi_k(r) &= e^{ik\cdot r} + f(k,\theta,\phi)\frac{e^{ikr}}{r} \\
    \lim_{r\to\infty} R_{kl}(r) &= \frac{A_l(k)e^{i\alpha}}{kr}\sin\left(kr - \frac{\pi l}{2} + \delta_l(k)\right) \\
    B_l(k) &= A_l(k)e^{i\alpha}
\end{align}
$B_l(k)$ is complex, whereas $A_l(k)$ is real. \\
(6.5) introduces the concept of the phase shift, which contains all the information surrounding the scattering.
(6.3) and (6.5) are physically equivalent equations.
\begin{align}
    f_l(k) &= \frac{2l+1}{k}e^{i\delta_l(k)}\sin\left(\delta_l(k)\right)
\end{align}
Where $f_l(k)$ is the amplitude of the $l$th partial wave. 
Then, this allows us to calculate the total scattering cross-section,
\begin{align}
    \sigma &= \sum_{l=0}^\infty \sigma_l \\
    \sigma_l &= \frac{4\pi(2l+1)}{k^2}\sin^2\left(\delta_l(k)\right)
\end{align}

Examples:
\begin{enumerate}
    \item $V(r) = 0 \implies$ no scattering 
        \begin{align}
            \delta_l &= 0 & f_l(k) &= 0 & \sigma & = 0
        \end{align}
    \item Scattering of a hard sphere, 
        \begin{equation}
            V(r) = \begin{cases} \infty & 0 \leq r \leq r_0 \\ 0 & r_0 < r < \infty \end{cases}
        \end{equation}
    \item Scattering of a potential well,
        \begin{equation}
            V(r) = \begin{cases} -V_0 & 0 \leq r \leq r_0 \\ 0 & r_0 < r < \infty \end{cases}
        \end{equation}
\end{enumerate}

So in general, by rewriting (6.5), the radial wavefunction has to have the form 
\begin{align}
    \lim_{r\to\infty} R_{kl}(r) &= \frac{\tilde{A}_l(k)e^{i\alpha}}{r} \left[\sin\left(kr - \frac{\pi l}{2}\right) + \tan(\delta_l(k))\cos\left(kr - \frac{\pi l}{2}\right)\right] \\
    \tan(\delta_l(k)) &= \frac{j_l(kr_0)}{n_l(kr_0)}
\end{align}
So for example 2 (an ISW), split into regions \RN{1} and \RN{2}.
\RN{1} is the well, \RN{2} is free space. 
So the solution in \RN{2} where $V_{\RN{2}} = 0$ is known.
\begin{align}
    R^\RN{2}(r)|_{r=r_0} &= R^\RN{1}(r)|_{r=r_0} \equiv 0 \\
    \frac{\p R^\RN{2}(r)}{\p r}\big|_{r_0} &= \frac{\p R^\RN{1}}{\p r}\big|_{r_0}
\end{align}
The evaluation of these boundaries at $r_0$ is what leads to $\tan(\delta_l(k))$ being the ratio of Bessel functions using $r_0$.
\begin{align}
    \sigma &= \sum_{l=0}^\infty \sigma_l \\
    \sigma_l &= \frac{4\pi(2l+1)}{k^2}\sin^2(\delta_l(k)) = \frac{4\pi(2l+1}{k^2}\frac{1}{1 + \cot^2(\delta_l(k))} \\
    \cot^2(\delta_l(k)) &= \frac{n^2_l(kr_0)}{j^2_l(kr_0)}
\end{align}
So what is the low-energy limit? I.e. as $k \to 0$.
\begin{align}
    \lim_{k\to0} \sigma_l(k) &= 4\pi(2l+1)\left(\frac{1}\lim_{k\to0}{k\cot(\delta_l(k))}\right)^2 \\
    \lim_{k\to0} k\cot(\delta_l(k)) &= \frac{kn_l(kr_0)}{j_l(kr_0)} \\
    \lim_{kr_0\to0} j_l(kr_0) &= \frac{(kr_0)^l}{(2l+1)!!} + O((kr_0)^{l+1}) \\
    \lim_{kr_0\to0} n_l(kr_0) &= -(2l-1)!!\frac{1}{(kr_0)^{l+1}} + O\left(\frac{1}{(kr_0)^l}\right)
\end{align}

\chapter{}
For any potential $V(r)$:
\begin{equation}
    \lim_{r\to\infty} R_{kl}(r) = \frac{\tilde{A}_{kl}(r)}{r}\left[\sin\left(kr-\frac{\pi l}{2}\right) + \tan(\delta_l(k))\cos\left(kr-\frac{\pi l}{2}\right)\right]
\end{equation}
For $V(r)$ as the hard sphere potential (a radial ISW, $0\leq r\leq r_0$),
\begin{equation}
    \tan(\delta_l(k)) = \frac{j_l(kr_0)}{n_l(kr_0)}
\end{equation}
We also know that
\begin{align}
    \sigma_l &= \frac{4\pi}{k^2}(2l+1)\sin^2(\delta_l(k)) \\
             &\equiv \frac{4\pi}{k^2}(2l+1)\frac{1}{1+\cot^2(\delta_l(k))}
\end{align}
So $\sigma_l$ is known for this potential.
\begin{itemize}
    \item We want to simplify this expression
    \item What happens for $l=0$ (s-wave)?
    \item What happens in the low energy limit as $k\to0$?
\end{itemize}
Start with the $k\to0$ limit, keeping $l$ general.
\begin{align}
    (2l+1)!! &= 1\times3\times5\times\dots\times(2l+1) \\
    \lim_{\rho\to0} j_l(\rho) &= \frac{\rho^l}{(2l+1)!!} + O(\rho^{l+1}) \\
    \lim_{\rho\to0} n_l(\rho) &= -(2l-1)!!\frac{1}{\rho^{l+1}} + O\left(\frac{1}{\rho^l}\right), (-1)!! = 1
\end{align}
For the hard sphere potential,
\begin{align}
    \lim_{k\to0} \tan(\delta_l(k)) &= -\frac{kr_0 (kr_0)^{2l}}{(2l-1)!!(2l+1)!!} + O((kr_0)^2) \\
    \sigma_l(k) &= \frac{4\pi(2l+1)}{k^2+\left(\frac{k}{\tan(\delta_l(k))}\right)^2} \\
    \lim_{k\to0} \sigma_l(k) &= \frac{4\pi(2l+1)}{0 + \left(\frac{k}{kr_0(kr_0)^{2l}}(2l-1)!!(2l+1)!!\right)^2} \\
                             &= \frac{4\pi(2l+1)r_0^2(kr_0)^{4l}}{\big((2l-1)!!(2l+1)!!\big)^2}
\end{align}
What can we say about this?
Firsly, $\sigma_l \propto r_0^2$, so dimension of length squared.
\textbf{Some sort of issue here with the powers of $\mathbf{kr_0}$ - stay tuned, kids.}

So what happens in the limit when $l\to0$ (s-wave)?
\begin{equation}
    \lim_{k\to 0} \sigma_0 = 4\pi\cdot r_0^2
\end{equation}
For classical cross-section off the sphere, no probabilities - you either are aimed into the sphere and hit it, or you aren't aimed at it and miss it. 
So the classical cross-section, 
\begin{equation}
    \sigma_{clsc} = \pi r_0^2,
\end{equation}
is just the cross-section of the sphere.
In the Quantum Mechanical case, the low energy limit is 4 times that of the classical limit due to interference effects.
\begin{equation}
    \lim_{k\to0} \sigma_0^{QM} = 4\sigma_0^{clsc}
\end{equation}
The Quantum Mechanical case appears to be the area of the entire curved surface of the sphere.

Consider the scattering length, $a = r_0$ - for hard sphere potential. 
In general, 
\begin{align}
    a &\equiv -\lim_{k\to0} \frac{\tan(\delta_{l=0}(k))}{k}
\end{align}
This will still be equal to $r_0$ for a hard sphere.
The cross-section will always go proportional to $a^2$,
\begin{equation}
    \sigma_0 \propto a^2
\end{equation}

\section{Resonances}
So in general for the l\textsuperscript{th}-wave, 
\begin{align}
    \sigma_l &= \frac{4\pi}{k^2}(2l+1)\frac{1}{1+\cot^2(\delta_l(k))}, ~0 \leq \cot^2(\delta_l(k)) \leq +\infty
\end{align}
If we fix $l$ and vary $k$ for $\sigma_l(k)$, where does $\sigma_l(k)$ reach a local maxima?
When $\cot(\delta_l(k)) = 0$ - this is called a \emph{resonance.}
Resonances are sharp peaks inbetween roughly uniform $\sigma_l(k)$ distributions.

\chapter{}
Last lecture summary:
\begin{itemize}
    \item Hard sphere potential 
        \begin{equation}
            \lim_{k\to0} k\cot(\delta_l(k)) = -\frac{(2l-1)!!(2l+1)!!}{r_0\cdot(kr_0)^{2l}}
        \end{equation}
    \item Cross-section
        \begin{equation}
            \sigma_l = \frac{4\pi(2l+1)}{k^2(1+\cot^2(\delta_l(k)))} \to \frac{4\pi(2l+1)r_0^2\cdot(kr_0)^{4l}}{[(2l-1)!!(2l+1)!!]^2}
        \end{equation}
    \item Factor of 4l suppresses with higher l
    \item $l=0$ - dominant contours(?) s-wave at $kr_0 < 1$.
    \item Scattering length, $l_s = a$
        \begin{align}
            a &= \lim_{k\to0} \left(-\frac{1}{k\cot(\delta_{l=0}(k))}\right) \\
            \lim_{k\to0} \sigma_0 &= 4\pi a^2
        \end{align}
\end{itemize}

\section{Resonances}
\begin{align}
    \sigma_l(k) &= \frac{4\pi}{k^2}(2l+1)\frac{1}{1+\cot^2(\delta_l(k))}
\end{align}
Hunting for $\sigma_l$'s maxima $\iff$ hunting for zeros of $\cot(\delta_l)$.
Can parameterise this:
\begin{align}
    \cot(\delta_l(k)) &= \frac{E_R - E}{\Gamma_R/2} \\
    \sigma_l(E) &= \frac{4\pi}{k^2}(2l+1)\frac{\Gamma_R^2/4}{(E-E_R)^2+\Gamma_R^2/4}
\end{align}
This forms a gaussian around $E_R$.
Maxima at $E_R$. 
For $\sigma_l^{max}/2$, $E=E_R \pm \Gamma_R/2$.
This is known as the Bret-Wigner resonance formula.
$\Gamma_R$ is the resonance width.

\section{The Optical Theorem}
\begin{align} 
    f(k,\theta,\cancel{\phi}) &= \sum_{l=0}^\infty f_l(k)P_l(\cos\theta) \\
    f_l(k) &= \frac{2l+1}{k}e^{i\delta_l(k)}\sin(\delta_l(k)) \\
    f(k,\theta=0) &= \sum_{l=0}^\infty \frac{2l+1}{k} e^{i\delta_l(k)}\sin(\delta_l(k))\cdot 1 \\
    \sigma_{tot} &= \sum_{l=0}^\infty \frac{4\pi}{k^2}(2l+1)\sin^2(\delta_l(k)) \\
    \text{Im}[f(k,\theta=0)] &= \sum_l \frac{2l+1}{k}\sin^2(\delta_l(k)) \\
    \sigma_{tot} &= \frac{4\pi}{k}\text{Im}[f(k,\theta=0)]
\end{align}

\section{Integral equation and Born approximation for scattering}
Central equation is the Schrodinger equation
\begin{equation}
    (\vec{\del}^2 + k^2)\psi_k(\vr) = U(\vr)\psi_k(\vr)
\end{equation}
Can write down a formal, general solution:
\begin{align}
    \psi(\vr) &= \psi^{(0)}(\vr) + \int G_0(k,\vr-\vr') U(\vr)\psi(\vr')\, d^3\vr' \\
    (\vec{\del}^2 &+ k^2)\psi^{(0)}(\vr) = 0 \\
    (\vec{\del}^2 &+ k^2)G_0(k,\vr-\vr') = \delta^{(3)}(\vr-\vr')
\end{align}
This is the definition of the Green's function, which is then used to find the general solution of the Schrodinger equation.

\chapter{}
\section{The Lippmann-Schwinger integral equation}
\begin{align}
    \psi_k(\vr) &= \psi_k^{(0)}(\vr) + \int G_0(\vec{k},\vr-\vr')U(\vr)\psi_k(\vr')d^3r' \\
    (\vec{\del}^2 + \vec{k}^2)\psi_k(\vr) &= U(\vr)\psi_k(\vr) \\
    \psi_k^{(0)}(r) &= e^{i\vec{k}\cdot\vr} \\
    (\vec{\del}^2 + \vec{k}^2)G_0(\vec{k},\vr) &= \delta^{(3)}(\vr)
\end{align}
Fourier transform of Green's function:
\begin{align}
    G_0(\vec{k},\vr) &= \int \frac{d^3q}{(2\pi)^3} e^{i\vec{q}\cdot\vr} \tilde{G}(\vec{k},\vec{q})\\
    (\vec{\del}^2 + \vec{k}^2)G_0(\vec{k},\vr) &= \int \frac{d^3q}{(2\pi)^3} (-q^2 + k^2)e^{i\vec{q}\cdot\vr}\tilde{G},~ \text{want } \tilde{G} = \frac{-1}{q^2-k^2} \\
                                               &= \int \frac{d^3q}{(2\pi)^3} e^{i\vec{q}\cdot\vr}\cdot 1 \equiv \delta^{(3)}(\vr),\text{ QED} \\
    G_0(\vec{k},\vr) &= \int \frac{d^3q}{(2\pi)^3} \frac{e^{i\vec{q}\cdot\vr}}{q^2-k^2} = -\frac{e^{i|\vec{k}||\vr|}}{4\pi |\vr|} \\
    G_0(\vec{k},\vr-\vr') &= -\frac{e^{i|k||\vr-\vr'|}}{4\pi |\vr-\vr'|}
\end{align}
Require $U(\vr')$ such that
\begin{equation}
    U(\vr') \begin{cases} \neq 0 & |\vr'| < d \\ \to 0 & |\vr'| > d \end{cases}
\end{equation}
Want to take $|\vr| >> d \implies \frac{\vr'}{\vr} << 1$.
Now Taylor expand these terms:
\begin{align}
    |\vr - \vr'| &= r\sqrt{1 - \left(\frac{\vr'}{\vr}\right)^2} \approx r\left(1-\frac12 2\frac{\vr\cdot\vr'}{r^2} + \cdots\right) \\
                 &\approx r - \frac{\vr\cdot\vr'}{r} = r - \hat{\vr}\cdot\vr' \\
    \vec{k}' &\equiv \hat{\vr}|\vec{k}| = \frac{\vr}{r}k
\end{align}
$\vec{k}'$ is the momentum of the scattered particle.
\begin{align}
    G_0(\vec{k},\vr-\vr') &\approx -\frac{1}{4\pi} \frac{e^{ikr}}{r} e^{-i\vec{k}'\cdot\vr'}
\end{align}
The general solution of the scattering equation therefore reads
\begin{align}
    \psi_k(\vr) &= e^{i\vec{k}\cdot\vr} - \frac{1}{4\pi} \int \frac{e^{ikr}}{r}e^{-i\vec{k}'\cdot\vr'}U(\vr')\psi_k(\vr)\,d^3r'
\end{align}
Can expand around solutions to the homogeneous solution in powers of the potential - Born expansion.
\begin{equation}
    f_{Born}(\vr) = -\frac{1}{4\pi}\int d^3r' ~ e^{i\vec{\Delta}\cdot\vr'}U(\vr'),~ \vec{\Delta} = \vec{k} - \vec{k}'
\end{equation}
$\vec{\Delta}$ is the momentum transfer.

\chapter{}
\begin{align}
    \psi_{Born, ~\vec{k}}(\vr) &= e^{i\vec{k}\cdot\vr} + \int G_0(\vr-\vr')U(\vr)e^{i\vec{k}\cdot\vr'}d^3r' \\
    G_0(\vr - \vr') &= \frac{e^{ikr}}{r}\cdot e^{-i\vec{k}'\cdot\vr'} = \frac{1}{r}e^{i(\vec{k}-\vec{k}')\cdot\vr'} \\
    f_B(\vr) &= -\frac{1}{4\pi}\int e^{i\vec{\Delta}\cdot\vr}U(r)\,d^3r \\
    \frac{d\sigma_B}{d\Omega} &= |f_B|^2 \\
    \sigma_B &= \int d\Omega\, |f_B|^2 \\
    \int d^3r &= \int_0^{2\pi} d\phi \, \int_0^\pi \sin\theta\,d\theta\, \ofnt r^2\,dr \\
    f_B &= -\frac{1}{4\pi} \underbrace{\int d\phi}_{2\pi} \ofnt r^2U(r)\,dr\, \int_0^\pi \sin\theta e^{i\Delta r\cos\theta}d\theta,~ h = \cos\theta \\
        &= -\frac12 \int dr\,r^2U(r)\, \int (-dh) e^{i\Delta rh} \\
        &= -\frac{1}{\Delta} \ofnt r\sin(\Delta r)U(r)\,dr
\end{align}
Gaussian potential,
\begin{align}
    U(r) \propto e^{-\alpha^2r^2}
\end{align}
Yukawa potential,
\begin{equation}
    U(r) \propto -\frac{1}{r}e^{-\mu r}
\end{equation}
In the limit of $\mu \to 0$, you get the Coulomb potential.\\
Perturbation theory is equivalent to the Born approximation.

\part{Statistical Mechanics}
\chapter{}
\begin{itemize}
    \item Complete description of a quantum system $\implies \psi(x) \iff$ a pure state
    \item An incomplete description of a quantum system $\iff$ a mixed state (do not know the full wave function)
\end{itemize}
The latter is Statistical quantum mechanics.
The physical system of interest is made of N different subsystems, $\alpha = 1,\dots,N \implies \psi^{(\alpha)}(x)$.
A useful tool/concept is the density matrix/operator, $\dens$.

Let's start with a pure state, and a subsystem $\alpha$.
\begin{align}
    \psi^{(\alpha)}(x) &= \langle x|\psi^\alpha\rangle = \langle x|\alpha\rangle \\
    \psi^\alpha &\equiv |\psi^\alpha\rangle \equiv |\alpha\rangle
\end{align}
A Hermitian (for this course, self-adjoint) operator,
\begin{align}
    \hat{O}^\dagger &= \hat{O} \\
    \hat{O}|n\rangle &= \lambda_n|n\rangle
\end{align}
The operator $\hat{O}$ describes a physical quantity and acts on states from the Hilbert spaces (wave functions) $|n\rangle$.
$|n\rangle$ is an eigen-vector of the operator $\hat{O}$. 
$\lambda_n \in \R$, an eigenvalue of $\hat{O}$.
In general, 
\begin{align}
    \hat{\Ham}|\psi\rangle &= |\tilde{\psi}\rangle \neq \lambda|\psi\rangle
\end{align}
However, there will exist states $|n\rangle$, such that,
\begin{equation}
    \hat{\Ham}|n\rangle = E_n|n\rangle,
\end{equation}
where $E_n$ is an energy of the state.
For any Hermitian operator, we can solve the eigenvalue relation, and $\lambda_n \in \R \forall n$. 
The set $\{|n\rangle\}$ must be complete.
\begin{align}
    \psi &= \sum_n c_n|n\rangle \\
    \sum_n &|n\rangle\langle n| = 1
\end{align}
This is equivalent to completeness of the set.
If the set is complete:
\begin{equation}
    \langle n|m\rangle = \delta_{nm}
\end{equation}
If the set is continuous:
\begin{equation}
    \langle n|m\rangle = \delta(n-m)
\end{equation}
These are conditions of orthonormality of the set.

\chapter{}
For the pure state, $|\psi\rangle$:
\begin{align}
    \langle A\rangle &= \langle\psi|\hat{A}|\psi\rangle = \sum_n\sum_m \underbrace{\langle n|\psi\rangle}_{c_n} \underbrace{\langle\psi|m\rangle}_{c_m^*} \underbrace{\langle m|\hat{A}|n\rangle}_{A_{mn}} \\
                     &= \sum_n\sum_m c_nc_m^* A_{mn}
\end{align}
So now consider a mixed state scenario, do not have $|\psi\rangle$.
The mixed state is a statiscal ensemble of pure states of subsystems for the mixed system state, with $|\alpha\rangle$s the subsystem wavefunctions. 
Within each subsystem $|\alpha\rangle$, we can compute 
\begin{equation}
    \langle A\rangle_\alpha \equiv \langle\alpha|\hat{A}|\alpha\rangle.
\end{equation}
So what about for the entire physical system?
\begin{equation}
    \langle A\rangle = \sum_\alpha W_\alpha \langle A\rangle_\alpha
\end{equation}
where $W_\alpha$ is the probability to find the measurement in the subsystem $|\alpha\rangle$.
The physical system is characterised by $\{|\alpha\rangle\}_\alpha,W_\alpha$.
For a system in a mixed state, the key quantity is the density operator, $\hat{\rho}$.
\begin{align}
    \hat{\rho} &= \sum_\alpha W_\alpha |\alpha\rangle \langle\alpha| = \sum_\alpha \rho_{\alpha\alpha} \\
    \langle A\rangle &= \sum_\alpha \langle\alpha|\hat{A}|\alpha\rangle \equiv \sum_\alpha \underbrace{W_\alpha |\alpha\rangle\langle\alpha|}_{\hat{\rho}_{\alpha\alpha}}\hat{A} \\
                     &\equiv \text{tr}(\hat{\rho}\hat{A})
\end{align}
Where $\langle A\rangle$ is the expectation value of $\hat{A}$ in a mixed state, and $\hat{\rho}$ is the density operator.
Density matrix:
\begin{align}
    \rho_{nm} &= \langle n|\hat{\rho}|m\rangle \\
    \langle A\rangle &= \sum_n \sum_m \underbrace{\sum_\alpha \langle n|\alpha\rangle W_\alpha \langle\alpha|m\rangle}_{\rho_{nm}} \underbrace{\langle m|\hat{A}|n\rangle}_{A_{mn}} \\
                     &= \sum_m \sum_n \rho_{nm}A_{mn} \equiv \text{tr}(\rho A)
\end{align}
Properties of density operator $\hat{\rho} \iff \rho_{nm}$ density matrix:
\begin{itemize}
    \item $\hat{\rho}^\dagger = \hat{\rho} \iff (\rho_{nm})^\dagger = \rho_{nm}$ - Hermitian
        \begin{equation}
            (\hat{\rho})^\dagger = \sum_\alpha |\alpha\rangle W_\alpha \langle\alpha| = \hat{\rho}
        \end{equation}
    \item Every Hermitian (self-adjoint) operator can be diagonalised. 
    \item We can find a complete orthonormal set, $\{|n\rangle\}_n$ where
        \begin{equation}
            \rho_{nm} = \begin{pmatrix} \rho_11 & 0 & \cdots \\ 0 & \rho_{22} & \\ & \vdots & 0 & \rho_{nm}\end{pmatrix} = \rho_{nn}\delta_{nm}, \rho_{nn} \in \R
        \end{equation}
    \item The physical meaning of $\rho_{nn}$ is the probability for the system to find itself in the state $|n\rangle$.
    \item For each $n$, $0 \leq \rho_{nn} \leq 1$, and $\sum_n \rho_{nn} = 1$
        \begin{equation}
            \implies \text{tr}(\hat{\rho}) = 1
        \end{equation}
    \item In the special case where the mixed state is the pure state, only $\rho_{n_0,n_0} = 1 \neq 0$
    \item For a truly mixed state, there will be a few $\rho_{nn} \neq 0, 0 \leq \rho_{nn} < 1$. Follows from the trace.
        \begin{equation}
            \text{tr}(\hat{\rho}^2) \begin{cases} < 1 & \text{mixed state} \\ = 1 & \text{pure state} \end{cases}
        \end{equation}
\end{itemize}

\part{Relativistic Quantum Mechanics}
\chapter{}
\begin{align}
    \hat{\Ham}\psi &= E\psi \\
    \hat{E} &= i\hbar\frac{\p}{\p t} 
\end{align}
Free case, $V = 0$:
\begin{align}
    \hat{\Ham}_{free} &= \frac{\hat{p}^2}{2m} \\
    \hat{p} &= -i\hbar\frac{\p}{\p x}
\end{align}
Free relativistic energy momentum relation:
\begin{align}
    E &= \vec{p}^2c^2 + m^2c^4 \\
    (\p_\mu\p^\mu &+ m^2)\Phi = 0
\end{align}
Klein Gordon equation - will use $\Phi$ instead of $\psi$ for wavefunction to show relativistic, $\hbar = c = 1$.
\begin{equation}
    \left(\p_\mu\p^\mu + m^2\frac{c^2}{\hbar^2}\right)\Phi = 0
\end{equation}

\chapter{}
\begin{equation}
    \left(\p_\mu\p^\mu + m^2\right)\Phi = 0
\end{equation}
Lorentz 4-vector:
\begin{align}
    x^\mu &= (t,\vec{x}), \mu = \underbrace{0}_{t},\underbrace{1,2,3}_{\vec{x}} \\
    x_\mu &= g_{\mu\nu}x^\nu \\
    x^\mu &= g^{\mu\nu}x_\nu \\
    g_{\mu\nu} &= \begin{pmatrix} +1 & 0 & 0 & 0 \\ 0 & -1 & 0 & 0 \\ 0 & 0 & -1 & 0 \\ 0 & 0 & 0 & -1 \end{pmatrix} = g^{\mu\nu} \\
    \p_\mu &= \frac{\p}{\p x^\mu} \\
    \p^\mu &= \frac{\p}{\p x_\mu} = g^{\mu\nu}\p_\nu \\
    \p_\mu\p^\mu &= \frac{\p}{\p x^\mu}\frac{\p}{\p x_\mu} = \frac{\p^2}{\p t^2} - \frac{\p^2}{\p\vec{x}^2} \\
    (\p_\mu\p^\mu &+ m^2)\Phi = 0
\end{align}
Klein-Gordon equation is Lorentz-covariant.
$\Phi$ is the relativistic wavefunction - \textit{more precisely, $\Phi$ is called the scalar field $\Phi(t,\vx)$. $\Phi$ has no indices so it is a scalar and it is a field as it depends on space and time.} \\
The Klein-Gordon equation is a free relativistic equation - there is no potential, and no other fields, only $\Phi$.
We want to introduce interactions of $\Phi(t,\vx)$ with $A_\mu(t,\vx)$, where $A_\mu$ is the vector field also known as the 4-vector potential.
\begin{align}
    A^\mu &= \begin{pmatrix} \frac{1}{c}\phi \\ \vec{A} \end{pmatrix} \equiv (\phi,\; \vec{A}) \\
    A_\mu &= g_{\mu\nu}A^\nu = (\phi,\;-\vec{A})
\end{align}
This is the electromagnetic vector field. \\
Interactions are introduced as follows:
\begin{align}
    \p_\mu \to D_\mu &\equiv \p_\mu - ieA_\mu(t,\vx)
\end{align}
$D_\mu$ here is the covariant derivative, $e$ is the absolute value of the electric charge, $e = -q$.
This is the \textbf{minimal way} to introduce interactions of $\Phi$ with electromagnetism.
Our modified Klein-Gordon equation becomes
\begin{align}
    (D^\mu D_\mu &+ m^2)\Phi = 0 \\
    D^\mu D_\mu &= (\p^\mu - ieA^\mu)(\p_\mu - ieA_\mu) \\
    E &\to E - q\phi \\
    \vec{p} &\to \vec{p} - q\vec{A}
\end{align}
Back to the free Klein-Gordon,
\begin{align}
    (\p_\mu\p^\mu &+ m^2)\Phi = 0
\end{align}
We want to define, $j^\mu$ - the 4-vector current.
\begin{equation}
    \p_\mu j^\mu = 0
\end{equation}
This is the continuity equation, or the current conservation equation.
\begin{align}
    j^\mu &= (c\rho,\;\vec{j}) \equiv (\rho,\;\vec{j}) \\
    \p_\mu j^\mu &= \p_t\rho + \p_i j^i = 0 \\
    j^\mu &= \frac{i}{2m}\left(\Phi^*\p^\mu\Phi - (\p^\mu\Phi)^*\Phi\right)
\end{align}
Check $\p_\mu j^\mu = 0$, if $\Phi$ satsifies Klein-Gordon.
\begin{align}
    \p_\mu j^\mu &= \frac{i}{2m}\left(\cancel{(\p_\mu\Phi^*)\p^\mu\Phi} + \Phi^*\p_\mu\p^\mu\Phi - (\p_\mu\p^\mu\Phi^*)\Phi - \cancel{(\p^\mu\Phi^*)(\p_\mu\Phi)} \right)
\end{align}
Now use the Klein-Gordon equation:
\begin{align}
    \p_\mu\p^\mu\Phi &= -m^2\Phi \\
    \p_\mu\p^\mu\Phi^* &= -m^2\Phi^* \\
    \p_\mu j^\mu &= \frac{i}{2m}\Big(\cancel{(\p_\mu\Phi^*)\p^\mu\Phi} + \underbrace{\Phi^*\p_\mu\p^\mu\Phi}_{-m^2\Phi} - \underbrace{(\p_\mu\p^\mu\Phi^*)\Phi}_{-m^2\Phi^*} - \cancel{(\p^\mu\Phi^*)(\p_\mu\Phi)} \Big) \\
    \p_\mu j^\mu &= 0
\end{align}
Interpretation: $j^\mu$ is the probability 4-current.
Specifically, taking $j^0 = \rho(t,\vx)$:
\begin{align}
    \int_V \rho(t,\vx)\,d^3\vx &= \text{Prob to find ptcl within V} \\
    \rho &= \frac{i}{2m}\left(\phi^*\p_t\Phi - (\p_t\Phi^*)\Phi\right)
\end{align}
This is not manifestly positive, so it cannot be a probability current.
Furthermore, there are $E > 0$ and $E < 0$ solutions to the Klein-Gordon equation.
\begin{align}
    \Phi_{special} &\approx e^{ip^\mu x_\mu},~ p^0 = +\sqrt{\vec{p}^2 + m^2} \\
    (\p_\mu\p^\mu &+ m^2)e^{ip^\mu x_\mu} = \left(-p^\mu p_\mu + m^2\right)e^{ip^\mu x_\mu} = 0 
\end{align}
This is a solution. 
\begin{align}
    \tilde{\Phi}_{special} &\approx e^{-ip^\mu x_\mu}
\end{align} 
This is also a solution.
\begin{align}
    E &= \pm \sqrt{\vec{p}^2 + m^2} \\
    p^\mu &= (E=p^0,\; \vec{p})
\end{align}

\chapter{}
\begin{align}
    E^2 &= p^2c^2 + m^2c^4 \\
    E^2 &- p^2 = m^2c^4 \\
    (\p_\mu\p^\mu &+ \frac{m^2c^2}{\hbar^2})\Phi(x,t) = 0 \\
    E &= \pm \sqrt{p^2c^2 + m^2c^4}
\end{align}
\section{Dirac Equation}
\begin{align}
    i\hbar\frac{\p\psi}{\p t} &= \Ham_D\psi(x,t) \\
                              &= -i\hbar c\sum_{k=1}^3 \alpha^k\frac{\p\psi}{\p x^k} + \beta mc^2\psi(x,t) \\
    \Ham_D &= \unl{\alpha}\cdot\unl{p}c + \beta mc^2,~ p_k = -i\hbar\del_k \\
    E^2\psi &= (p^2c^2 + m^2c^4)\psi \\
    \Ham_D^2 &= (-\hbar^2c^2\del^2 + m^2c^4) \\
             &= (-i\hbar c\alpha^j\p_j + \beta mc^2)(-i\hbar c\alpha^k\p_k + \beta mc^2) \\
             &= -\hbar^2c^2\alpha^j\alpha^k\p_j\p_k - i\hbar c^3m\alpha^i\beta\p_i - i\hbar c^3m\beta\alpha^k\p_k + \beta^2m^2c^4 \\
             &= -\frac{\hbar^2c^2}{2}\left\{\alpha^j,\alpha^k\right\}\p_j\p_k - i\hbar c^3m\{\alpha^i,\beta\} + \frac{m^2c^4}{2}\{\beta,\beta\}  \\
    \{\alpha^j,\alpha^k\} &= \alpha^j\alpha^k + \alpha^k\alpha^j
\end{align}
Klein-Gordon is fulfilled if
\begin{align}
    \{\alpha^j,\alpha^k\} &= 2\delta^{jk} \\
    \{\alpha^i,\beta\} &= 0 \\
    \{\beta,\beta\} &= 2
\end{align}
These conditions define what is called \textbf{Dirac algebra.}
Dirac algebra is satisfied by choice e.g.
\begin{align}
    \alpha^k &= \begin{pmatrix} 0 & \sigma^k \\ \sigma^k & 0 \end{pmatrix} & \beta &= \begin{pmatrix} I_2 & 0 \\ 0 & -I_2\end{pmatrix} \\
    \alpha^1 &= \begin{pmatrix} 0 & 0 & 0 & 1 \\ 0 & 0 & 1 & 0 \\ 0 & 1 & 0 & 0 \\ 1 & 0 & 0 & 0 \end{pmatrix} & \beta &= \begin{pmatrix} 1 & 0 & 0 & 0 \\ 0 & 1 & 0 & 0 \\ 0 & 0 & -1 & 0 \\ 0 & 0 & 0 & -1 \end{pmatrix}
\end{align}
$\sigma^k$ denotes the k-th Pauli matrix.
Consequences:
\begin{align}
    \psi = \begin{pmatrix} \psi_1 \\ \psi_2 \\ \psi_3 \\ \psi_4\end{pmatrix}
\end{align}
This is known as the \textbf{Dirac spinor.}
Hermiticity of $\Ham_D$
\begin{align}
    (\alpha^k)^\dagger &= \alpha^k & \beta^\dagger &= \beta 
\end{align}
$\alpha$ and $\beta$ are not unique:
\begin{align}
    \tilde{\alpha}^k &= M\alpha^kM^{-1} & \tilde{\beta} &= M\beta M^{-1}
\end{align}
Requiring $M$ to be Hermitian.

Let's define
\begin{align}
    \rho &= \psi^\dagger\psi \\
         &= \begin{pmatrix} \psi_1^* & \psi_2^* & \psi_3^* & \psi_4^*\end{pmatrix}\begin{pmatrix} \psi_1 \\ \psi_2 \\ \psi_3 \\ \psi_4\end{pmatrix} \\
         &= |\psi_1|^2 + |\psi_2|^2 + |\psi_3|^2 + |\psi_4|^2 \leq 0 \\
    \frac{\p\rho}{\p t} &= \psi^\dagger\frac{\p\psi}{\p t} + \frac{\p\psi^\dagger}{\p t}\psi \\
    \psi^\dagger\frac{\p\psi}{\p t} &= \psi^\dagger \frac{1}{i\hbar}\Ham_D\psi \\
                                    &= \psi^\dagger \left(-c\alpha^k\del_k + \frac{mc^2}{i\hbar}\beta\right)\psi \\
                                    &= -c\psi^\dagger\unl{\alpha}\cdot\unl{\del}\psi + \frac{mc^2}{i\hbar}\psi^\dagger \beta\psi \\
    \frac{\p\psi^\dagger}{\p t}\psi &= -c(\unl{\del}\psi^\dagger)\cdot\unl{\alpha}\psi - \frac{mc^2}{i\hbar}\psi^\dagger\beta\psi \\ 
    \frac{\p\rho}{\p t} &= -c(\del\psi^\dagger)\cdot \unl{\alpha}\psi - c\psi^\dagger\unl{\alpha} \cdot\unl{\del}\psi \\ 
    \frac{\p\rho}{\p t} &+ \unl{\del}\cdot(\psi^\dagger c\unl{\alpha}\psi) = 0 \\
    \frac{\p\rho}{\p t} &+ \unl{\del}\cdot\unl{j} = 0 \\
    \unl{j} &= \psi^\dagger c\unl{\alpha}\psi
\end{align}
Covariant form:
\begin{align}
    \{\beta,\beta\} = 2 &\implies \beta^2 I_4 \\
    i\hbar\frac{\p\psi}{\p t} &= -i\hbar c\alpha^k \frac{\p\psi}{\p x^k} + \beta mc^2\psi \\
    -i\beta\frac{1}{c}\frac{\p\psi}{\p t} &+ i\beta\alpha^k\frac{\p\psi}{\p x^k} - \frac{mc}{\hbar}\psi \\
    \p_\mu &= \left(\frac{1}{c}\frac{\p}{\p t},\frac{\p}{\p x^1},\frac{\p}{\p x^2},\frac{\p}{\p x^3}\right) \\
    \left(-i\gamma^\mu\p_\mu + \frac{mc}{\hbar}\right)\psi &= 0
\end{align}
Introduced two new 4x4 matrices
\begin{align}
    \gamma^0 &= \beta & \gamma^k &= \beta\alpha^k
\end{align}
Setting $\hbar = c = 1$
\begin{equation}
    (i\cancel{\p} - m)\psi = 0
\end{equation}
\begin{align}
    \gamma^0 &= \begin{pmatrix} 0 & I_2 \\ I_2 & 0 \end{pmatrix} & \gamma^k &= \begin{pmatrix} 0 & \sigma^k \\ -\sigma^k & 0 \end{pmatrix}
\end{align}

\chapter{}
Klein-Gordon equation:
\begin{align}
    (\p_m\p^m - m^2)\Phi(x^\mu) = 0
\end{align}
This is manifestly Lorentz-covariant.\\
Dirac equation:
\begin{align}
    (i\gamma^\mu\p_\mu - m)\Psi(x) &= 0 \\
    \{\gamma^\mu,\gamma^\nu\} &= 2g^{\mu\nu}
\end{align}
$\Psi(x)$ is a 4-component spinor - a bispinor in $3+1$ dimensions.

Problems with interpretation of $\Phi(x)$ as the wavefunction:
\begin{itemize}
    \item $E^2 = \vec{p}^2 + m^2 \implies E = \pm \sqrt{\vec{p}^2+m^2}$, so positive and negative energies are both present, making an unstable solution. 
    \item Define $j^\mu$
        \begin{equation}
            \p_\mu j^\mu = 0
        \end{equation}
        This is a conserved current.
        \begin{align}
            j^0(x) = \rho(x)
        \end{align}
        But this is not manifestly positive.
        Probability, P to find particle:
        \begin{align}
            P &= \int_V \rho(x)\,d^3x
        \end{align}
        So the quantum mechanical meaning of $\rho(x)$ is the probablity density.
\end{itemize}
Common origin of these problems is that the Klein-Gordon equation is second order in derivatives with respect to $\p_0 = \frac{\p}{\p t}$.
\begin{align}
    \rho(x) &\approx \Phi^*\p_t \Phi - (\p_t\Phi^*)\Phi \\
    \p_\mu j^\mu &= 0 \to \p_t\rho + \cdots
\end{align}
Dirac's original idea: Find Lorentz covariant equation which is first order in $\p_t$.
\begin{align}
    (i\gamma^\mu\p_\mu - m)\Psi(x) &= 0
\end{align}
$\Psi(x)$ is a relativistic field, bispinor.
\begin{align}
    \Psi(x) &= \begin{pmatrix} \psi^1(x) \\ \psi^2(x) \\ \psi^3(x) \\ \psi^4(x) \end{pmatrix}
\end{align}
Normal $\to$ Weyl spinor $\to$ 2 components. Normal spinors are 2-component ones. \\
Dirac spinor, $\Psi(x) \to$ 4 components (bispinor).

The Klein-Gordon equation is for $\Phi(x)$, a scalar field - a one-component object, for spin $=0$ particles. 
Bosons are particles with integer spin, i.e. spin $=0$, e.g. Higgs boson.

The Dirac equation is for the four-component $\Psi(x)$, which describes fermions, particles of half-integer spin, lowest of these is $\frac12$.
Why 4-components then?
Spin-$\frac12$ particles, spin aligned along the direction of momentum vector positively or negatively.
Spinor has four-components as it did not solve the energy problem. 
There are still positive and negative energies, each of them has spin $\pm\frac12$, i.e. four states.\\
In Klein-Gordon, $\Phi(x)$ is complex, so two degrees of freedom.
$\Phi(x)$ and $\Psi(x)$ are not wavefunctions, they are operators acting on the states in Hilbert space.
Positive energy part creates the particle, the negative energy destroys the particle.

Let's rederive (very compactly) the Dirac equation.
Assume the Dirac equation and relate it to the Klein Gordon equation.
\begin{align}
    (\p_\mu\p^\mu + m^2)\Phi &= 0 \\
    (\p_0^2 + m^2 - \p_{\vx}^2)\Phi &= 0 \\
    (a^2 - b^2)\Phi &= 0 \\
    (a \pm b)\Phi &= 0 
\end{align}
If we can apply (4.13) to Klein-Gordon, then (4.14) would be a simple 'Dirac' equation.
But $\p_0^2 + m^2 \neq a^2$, so must use $\gamma$ matrices to make this work. 
\begin{align}
    (i\gamma^\nu\p_\nu + m)(i\gamma^\mu\p_\mu - m)\Psi &= 0 \\ 
    \{\gamma^\mu,\gamma^\nu\} &= 2g^{\mu\nu}
\end{align}
(4.16) is the defining equation for $\gamma$ matrices and is known as Clifford algebra.
\begin{align}
    (-\gamma^\nu\gamma^\mu\p_\nu\p_\mu - \cancel{mi\gamma^\nu\p_\nu} + \cancel{im\gamma^\mu\p_\mu} - m^2)\Psi &= 0 \\
    (-\gamma^\nu\gamma^\mu\p_\nu\p_\mu - m^2)\Psi &= 0 \\
    \gamma^\nu\gamma^\mu \p_\nu\p_\mu &= \frac12\gamma^\nu\gamma^\mu(\p_\nu\p_\mu + \p_\mu\p_\nu),~ \gamma^\mu\p_\mu = \fp \\
                                      &= \frac12\gamma^\nu\gamma^\mu\p_\nu\p_\mu + \frac12\gamma^\nu\gamma^\mu\p_\mu\p_\nu \\
                                      &= \frac12\left(\{\gamma^\mu,\gamma^\nu\}\right)\p_\nu\p_\mu = \frac12\times 2g^{\mu\nu}\p_\nu\p_\mu \\
                                      &= \p_\mu\p^\mu \\
    -(\p_\mu\p^\mu + m^2)\Psi &= 0
\end{align}
This is the Klein-Gordon equation from the Dirac equation.
So any solution to the Dirac equation will satisfy the Klein-Gordon equation. 
This explains why we didn't solve the energy sign problem.

\chapter{}

\chapter{}
Last lecture, discussed Lorentz transformations of Dirac spinors and Dirac equation.
\begin{align}
    x\to x'^\mu&\equiv \Lambda^\mu_\nu x^\nu \\
    \Psi(x) \to \Psi'(x')&\equiv S(\Lambda)\Psi(x) \\
    (i\cancel{\p}-m)\Psi(x) &= 0 \\
    (i\gamma^\mu\p_\mu - m)\Psi(x) &= 0 \\
    (i\gamma^\mu\p_\mu' -m)\Psi'(x') &= 0 \\
    S^{-1}(\Lambda)\gamma^\nu S(\Lambda) &= \gamma^\mu\Lambda^\nu_\mu \\
    \bar{\Psi} &\equiv \Psi^\dagger\gamma^0
\end{align}
This is the Dirac conjugate spinor.
\begin{align}
    \bar{\Psi}'(x') &= \bar{\Psi}(x)\cdot S^{-1}(\Lambda) \\
    \bar{\Psi}\Psi &= \text{Lorentz singlet/scalar}
\end{align}
Lorentz singlets do not transform under Lorentz transformations.
\begin{align}
    \bar{\Psi}\gamma^\mu\Psi &= \text{Lorentz vector} \\
    \bar{\Psi}'\gamma^\mu\Psi &\to \bar{\Psi}S^{-1}\gamma^\mu S\Psi = \Lambda^\mu_\nu \bar{\Psi}\gamma^\nu\Psi \\
    \bar{\Psi}\gamma^\mu\Psi &= \Lambda^\mu_\nu(\bar{\Psi}\gamma^\nu\Psi)
\end{align}
Dirac Langrangian:
\begin{align}
    \La_D &= \bar{\Psi}(i\gamma^\mu\p_\mu - m)\Psi
\end{align}

\section{General Solutions to Klein Gordon}
\begin{align}
    (\p^\mu\p_\mu &+ m^2)\Phi(x) = 0 \\
    \Phi(x) &= \int \frac{d^3p}{(2\pi)^3} \frac{1}{2w_p}\left(a_{\vp} e^{-ip^\mu x_\mu} + b^{\dagger}_{\vp} e^{+ip^\mu x_\mu}\right)\Bigg|_{p^0\equiv w_p\equiv +\sqrt{m^2+\vp^2}} \\
    p^\mu &= (p^0\equiv w_{\vp},\; \vp)
\end{align}
We integrate over $\vp$.
\begin{align}
    (\p^\mu\p_\mu + m^2)e^{\pm ip^\mu x_\mu} &= (-p^\mu p_\mu + m^2)e^{\pm ip^\mu x_\mu} \\
    m^2 - p^\mu p_\mu &= 0,~ p^0 = w_{\vp}
\end{align}
$p^0 = w_{\vp}$ is on-mass shell condition.\\
$a_{\vp}$ is an arbitrary function of $\vp$, using its complex conjugate and not another arbitrary function ensures $\Phi(x)$ is real.
\begin{align}
    \int \frac{d^3\vp}{(2\pi)^3} \frac{1}{2w_{\vp}},~ w_{\vp} = \sqrt{\vp^2 + m^2}
\end{align}
This is a Lorentz-invariant integration measure. Why? \\
Manifestly Lorentz-invariant measure:
\begin{align}
    \int \frac{d^4p}{(2\pi)^3} \delta^{(1)}(p_\mu p^\mu - m^2) &= \sum_{\text{solutions}} \int \frac{d^4p}{(2\pi)^3} \frac{\delta(p^0=\text{solution}}{|\p_{p_0} (p^\mu p_\mu - m^2)|} \\
                                                               &= \int \frac{d^3\vp}{(2\pi)^3} \left(\frac{1}{2w_{\vp}}\bigg|_{p_0 = +w_p} + \frac{1}{2w_{\vp}}\bigg|_{p_0=-w_p}\right)
\end{align}
So (6.19) is Lorentz invariant as it follows from simplifying something manifestly Lorentz invariant.
\begin{equation}
    p_0 = w_{\vp} = +\sqrt{\vp^2 + m^2} 
\end{equation}
Is a positive relativistic energy. 
\begin{align}
    e^{-ip^\mu x_\mu} &= e^{-iw_{\vp}t}\cdot e^{+i\vp\vx} \\
    e^{+ip^\mu x_\mu} &= e^{+iw_{\vp}t}\cdot e^{-i\vp\vx} \\
\end{align}
Which gives a negative energy solution in the second line.
So $\Phi(x)$ is actually an operator, $\hat{\Phi}(x)$, so are $a_p$ and $b^\dagger$.
\begin{itemize}
    \item $\hat{a}_{\vp}$ is the annihilation operator of a particle, 'a'.
    \item $\hat{a}^{\dagger}_{\vp}$ is the creation operator of a particle, 'a'.
    \item $\hat{b}_{\vp}$ is the creation operator of a particle, 'b'.
    \item $\hat{b}^{\dagger}_{\vp}$ is the annihilation operator of a particle, 'b'.
\end{itemize}
a and b are antiparticles of each other.

\textbf{One or two lectures may be missing at the end here.}















\end{document}
